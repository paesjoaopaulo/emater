% RESUMO--------------------------------------------------------------------------------

\begin{resumo}[RESUMO]
\begin{SingleSpacing}

% Não altere esta seção do texto--------------------------------------------------------
\imprimirautorcitacao. \imprimirtitulo. \imprimirdata. \pageref {LastPage} f. \imprimirprojeto\ – \imprimirprograma, \imprimirinstituicao. \imprimirlocal, \imprimirdata.\\
%---------------------------------------------------------------------------------------

A presente proposta refere-se ao desenvolvimento de uma aplicação web para o cálculo de correção e equilíbrio do solo. A aplicação será desenvolvida com a técnica de desenvolvimento ágil, TDD e será modelada com diagramas da UML, que fornecerá as visões de funcionalidades, organização de classes e implantação, além de fornecer uma visão do modelo de banco de dados com o DER. O desenvolvimento será baseado, principalmente, nas linguagens de programação PHP e JavaScript, sendo o Laravel o \textit{framework} PHP para para o \textit{back-end} e o ReactJS a biblioteca JavaScript para o desenvolvimento do \textit{front-end} em conjunto com o Ant Design for React, além de empregar o MySQL como SGBD. O Codeception será o \textit{framework} utilizado para a escrita e execução dos testes automatizados nas etapas \textit{red} e \textit{green} do TDD. O Git será usado como repositório de código e este será hospedado no GitHub. O gerenciamento das tarefas será feito com o Trello, que proporcionará, de forma visual, uma visão geral do andamento do desenvolvimento do projeto utilizando o conceito de Kanban.

\textbf{Palavras-chave}: Correção e equilíbrio do solo. Desenvolvimento de sistema. Desenvolvimento web.

\end{SingleSpacing}
\end{resumo}

% OBSERVAÇÕES---------------------------------------------------------------------------
% Altere o texto inserindo o Resumo do seu trabalho.
% Escolha de 3 a 5 palavras ou termos que descrevam bem o seu trabalho 
