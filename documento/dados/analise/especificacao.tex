\subsection{Especificação de Requisitos}
\label{sec:titSecEspReq}

Texto...

\begin{quadro}[H]
    \begin{tabular}{|p{3cm}|p{11cm}|}
        \hline
        \textbf{RS300} & \textbf{Cadastrar dados para a correção do cálcio e magnésio no solo}                  \\
        \hline
        Sumário        & O sistema deve permitir o usuário informar dados acerca da correção do cálcio no solo. \\
        \hline
        Pré-condições  & O usuário deve estar logado no sitema                                                  \\
        \hline
        Atores         & Usuário                                                                                \\
        \hline
        Descrição      &
        \begin{itemize}
            \item Informa o percentual de cálcio desejado na CTC
            \item Seleciona qual fonte de cálcio será utilizada na correção
            \item Informa o custo médio em R\$/ha do corretivo
            \item Informa o percentual de PRNT do corretivo
            \item Informa o teor de CaO do corretivo
        \end{itemize}                                                                               \\
        \hline
        Alternativas   &
        \begin{itemize}
            \item Caso o usuário não saiba o teor de CaO, será utilizado o valor médio do corretivo
        \end{itemize}                                                                               \\
        \hline
        Exceção        &
        \begin{itemize}
            \item Caso o usuário informe um valor percental fora do intervalo de 0 a 100, não poderá proseguir com o preenchimento de outros campos.
            \item Caso o usuário informa um corretivo que não exista, uma mensagem de erro deverá ser exibida.
        \end{itemize}                                                                               \\
        \hline
    \end{tabular}
\end{quadro}

\begin{quadro}[H]
    \begin{tabular}{|p{3cm}|p{11cm}|}
        \hline
        \textbf{RS301} & \textbf{Alterar dados para a correção do cálcio e magnésio no solo}          \\
        \hline
        Sumário        & O sistema deve permitir o usuário alterar os dados sobre a correção do solo. \\
        \hline
        Pré-condições  & \begin{itemize}
            \item O usuário deve estar logado
            \item O usuário deverá acessar uma ??análise?? já existente
        \end{itemize}                                                    \\
        \hline
        Atores         & Usuário                                                                      \\
        \hline
        Descrição      &
        \begin{itemize}
            \item Informa o percentual de cálcio desejado na CTC
            \item Seleciona qual fonte de cálcio será utilizada na correção
            \item Informa o custo médio em R\$/ha do corretivo
            \item Informa o percentual de PRNT do corretivo
            \item Informa o teor de CaO do corretivo
        \end{itemize}                                                                     \\
        \hline
        Alternativas   &
        \begin{itemize}
            \item Caso o usuário não saiba o teor de CaO, será utilizado o valor médio do corretivo
        \end{itemize}                                                                     \\
        \hline
        Exceção        &
        \begin{itemize}
            \item Caso o usuário informe um valor percental fora do intervalo de 0 a 100, não poderá proseguir com o preenchimento de outros campos.
            \item Caso o usuário informe um corretivo que não exista, uma mensagem de erro deverá ser exibida.
        \end{itemize}                                                                     \\
        \hline
    \end{tabular}
\end{quadro}

\begin{quadro}[H]
    \begin{tabular}{|p{3cm}|p{11cm}|}
        \hline
        \textbf{RS302} & \textbf{Alterar dados para a correção do cálcio e magnésio no solo}          \\
        \hline
        Sumário        & O sistema deve permitir o usuário alterar os dados sobre a correção do solo. \\
        \hline
        Pré-condições  & \begin{itemize}
            \item O usuário deve estar logado
            \item O usuário deverá acessar uma ??análise?? já existente
        \end{itemize}                                                    \\
        \hline
        Atores         & Usuário                                                                      \\
        \hline
        Descrição      &
        \begin{itemize}
            \item Informa o percentual de cálcio desejado na CTC
            \item Seleciona qual fonte de cálcio será utilizada na correção
            \item Informa o custo médio em R\$/ha do corretivo
            \item Informa o percentual de PRNT do corretivo
            \item Informa o teor de CaO do corretivo
        \end{itemize}                                                                     \\
        \hline
        Alternativas   &
        \begin{itemize}
            \item Caso o usuário não saiba o teor de CaO, será utilizado o valor médio do corretivo
        \end{itemize}                                                                    \\
        \hline
        Exceção        &
        \begin{itemize}
            \item Caso o usuário informe um valor percental fora do intervalo de 0 a 100, não poderá proseguir com o preenchimento de outros campos.
            \item Caso o usuário informe um corretivo que não exista, uma mensagem de erro deverá ser exibida.
        \end{itemize}                                                                    \\
        \hline
    \end{tabular}
\end{quadro}

\begin{quadro}[H]
    \begin{tabular}{|p{3cm}|p{11cm}|}
        \hline
        \textbf{RS303} & \textbf{Alterar dados para a correção do cálcio e magnésio no solo}          \\
        \hline
        Sumário        & O sistema deve permitir o usuário alterar os dados sobre a correção do solo. \\
        \hline
        Pré-condições  & \begin{itemize}
            \item O usuário deve estar logado
            \item O usuário deverá acessar uma ??análise?? já existente
        \end{itemize}                                                   \\
        \hline
        Atores         & Usuário                                                                      \\
        \hline
        Descrição      &
        \begin{itemize}
            \item Informa o percentual de cálcio desejado na CTC
            \item Seleciona qual fonte de cálcio será utilizada na correção
            \item Informa o custo médio em R\$/ha do corretivo
            \item Informa o percentual de PRNT do corretivo
            \item Informa o teor de CaO do corretivo
        \end{itemize}                                                                    \\
        \hline
        Alternativas   &
        \begin{itemize}
            \item Caso o usuário não saiba o teor de CaO, será utilizado o valor médio do corretivo
        \end{itemize}                                                                    \\
        \hline
        Exceção        &
        \begin{itemize}
            \item Caso o usuário informe um valor percental fora do intervalo de 0 a 100, não poderá proseguir com o preenchimento de outros campos.
            \item Caso o usuário informe um corretivo que não exista, uma mensagem de erro deverá ser exibida.
        \end{itemize}                                                                    \\
        \hline
    \end{tabular}
\end{quadro}

\begin{quadro}[H]
    \begin{tabular}{|p{3cm}|p{11cm}|}
        \hline
        \textbf{RS304} & \textbf{Alterar dados para a correção do cálcio e magnésio no solo}          \\
        \hline
        Sumário        & O sistema deve permitir o usuário alterar os dados sobre a correção do solo. \\
        \hline
        Pré-condições  & \begin{itemize}
            \item O usuário deve estar logado
            \item O usuário deverá acessar uma ??análise?? já existente
        \end{itemize}                                                   \\
        \hline
        Atores         & Usuário                                                                      \\
        \hline
        Descrição      &
        \begin{itemize}
            \item Informa o percentual de cálcio desejado na CTC
            \item Seleciona qual fonte de cálcio será utilizada na correção
            \item Informa o custo médio em R\$/ha do corretivo
            \item Informa o percentual de PRNT do corretivo
            \item Informa o teor de CaO do corretivo
        \end{itemize}                                                                    \\
        \hline
        Alternativas   &
        \begin{itemize}
            \item Caso o usuário não saiba o teor de CaO, será utilizado o valor médio do corretivo
        \end{itemize}                                                                    \\
        \hline
        Exceção        &
        \begin{itemize}
            \item Caso o usuário informe um valor percental fora do intervalo de 0 a 100, não poderá proseguir com o preenchimento de outros campos.
            \item Caso o usuário informe um corretivo que não exista, uma mensagem de erro deverá ser exibida.
        \end{itemize}                                                                    \\
        \hline
    \end{tabular}
\end{quadro}

\begin{quadro}[H]
    \begin{tabular}{|p{3cm}|p{11cm}|}
        \hline
        \textbf{RS305} & \textbf{Alterar dados para a correção do cálcio e magnésio no solo}          \\
        \hline
        Sumário        & O sistema deve permitir o usuário alterar os dados sobre a correção do solo. \\
        \hline
        Pré-condições  & \begin{itemize}
            \item O usuário deve estar logado
            \item O usuário deverá acessar uma ??análise?? já existente
        \end{itemize}                                                   \\
        \hline
        Atores         & Usuário                                                                      \\
        \hline
        Descrição      &
        \begin{itemize}
            \item Informa o percentual de cálcio desejado na CTC
            \item Seleciona qual fonte de cálcio será utilizada na correção
            \item Informa o custo médio em R\$/ha do corretivo
            \item Informa o percentual de PRNT do corretivo
            \item Informa o teor de CaO do corretivo
        \end{itemize}                                                                    \\
        \hline
        Alternativas   &
        \begin{itemize}
            \item Caso o usuário não saiba o teor de CaO, será utilizado o valor médio do corretivo
        \end{itemize}                                                                    \\
        \hline
        Exceção        &
        \begin{itemize}
            \item Caso o usuário informe um valor percental fora do intervalo de 0 a 100, não poderá proseguir com o preenchimento de outros campos.
            \item Caso o usuário informe um corretivo que não exista, uma mensagem de erro deverá ser exibida.
        \end{itemize}                                                                    \\
        \hline
    \end{tabular}
\end{quadro}

\begin{quadro}[H]
    \begin{tabular}{|p{3cm}|p{11cm}|}
        \hline
        \textbf{RS306} & \textbf{Alterar dados para a correção do cálcio e magnésio no solo}          \\
        \hline
        Sumário        & O sistema deve permitir o usuário alterar os dados sobre a correção do solo. \\
        \hline
        Pré-condições  & \begin{itemize}
            \item O usuário deve estar logado
            \item O usuário deverá acessar uma ??análise?? já existente
        \end{itemize}                                                   \\
        \hline
        Atores         & Usuário                                                                      \\
        \hline
        Descrição      &
        \begin{itemize}
            \item Informa o percentual de cálcio desejado na CTC
            \item Seleciona qual fonte de cálcio será utilizada na correção
            \item Informa o custo médio em R\$/ha do corretivo
            \item Informa o percentual de PRNT do corretivo
            \item Informa o teor de CaO do corretivo
        \end{itemize}                                                                    \\
        \hline
        Alternativas   &
        \begin{itemize}
            \item Caso o usuário não saiba o teor de CaO, será utilizado o valor médio do corretivo
        \end{itemize}                                                                    \\
        \hline
        Exceção        &
        \begin{itemize}
            \item Caso o usuário informe um valor percental fora do intervalo de 0 a 100, não poderá proseguir com o preenchimento de outros campos.
            \item Caso o usuário informe um corretivo que não exista, uma mensagem de erro deverá ser exibida.
        \end{itemize}                                                                    \\
        \hline
    \end{tabular}
\end{quadro}

\begin{quadro}[H]
    \begin{tabular}{|p{3cm}|p{11cm}|}
        \hline
        \textbf{RS307} & \textbf{Exibir o percentual de cálcio ideal no solo}                         \\
        \hline
        Sumário        & O sistema deve permitir o usuário alterar os dados sobre a correção do solo. \\
        \hline
        Pré-condições  & \begin{itemize}
            \item O usuário deve estar logado
            \item O usuário deverá acessar uma ??análise?? já existente
        \end{itemize}                                                   \\
        \hline
        Atores         & Usuário                                                                      \\
        \hline
        Descrição      &
        \begin{itemize}
            \item Informa o percentual de cálcio desejado na CTC
            \item Seleciona qual fonte de cálcio será utilizada na correção
            \item Informa o custo médio em R\$/ha do corretivo
            \item Informa o percentual de PRNT do corretivo
            \item Informa o teor de CaO do corretivo
        \end{itemize}                                                                    \\
        \hline
        Alternativas   &
        \begin{itemize}
            \item Caso o usuário não saiba o teor de CaO, será utilizado o valor médio do corretivo
        \end{itemize}                                                                    \\
        \hline
        Exceção        &
        \begin{itemize}
            \item Caso o usuário informe um valor percental fora do intervalo de 0 a 100, não poderá proseguir com o preenchimento de outros campos.
            \item Caso o usuário informe um corretivo que não exista, uma mensagem de erro deverá ser exibida.
        \end{itemize}                                                                    \\
        \hline
    \end{tabular}
\end{quadro}

\begin{quadro}[H]
    \begin{tabular}{|p{3cm}|p{11cm}|}
        \hline
        \textbf{RS308} & \textbf{Exibir teor de magnésio atualmente no solo}                          \\
        \hline
        Sumário        & O sistema deve permitir o usuário alterar os dados sobre a correção do solo. \\
        \hline
        Pré-condições  & \begin{itemize}
            \item O usuário deve estar logado
            \item O usuário deverá acessar uma ??análise?? já existente
        \end{itemize}                                                   \\
        \hline
        Atores         & Usuário                                                                      \\
        \hline
        Descrição      &
        \begin{itemize}
            \item Informa o percentual de cálcio desejado na CTC
            \item Seleciona qual fonte de cálcio será utilizada na correção
            \item Informa o custo médio em R\$/ha do corretivo
            \item Informa o percentual de PRNT do corretivo
            \item Informa o teor de CaO do corretivo
        \end{itemize}                                                                    \\
        \hline
        Alternativas   &
        \begin{itemize}
            \item Caso o usuário não saiba o teor de CaO, será utilizado o valor médio do corretivo
        \end{itemize}                                                                    \\
        \hline
        Exceção        &
        \begin{itemize}
            \item Caso o usuário informe um valor percental fora do intervalo de 0 a 100, não poderá proseguir com o preenchimento de outros campos.
            \item Caso o usuário informe um corretivo que não exista, uma mensagem de erro deverá ser exibida.
        \end{itemize}                                                                    \\
        \hline
    \end{tabular}
\end{quadro}

\begin{quadro}[H]
    \begin{tabular}{|p{3cm}|p{11cm}|}
        \hline
        \textbf{RS309} & \textbf{Exibir o percentual ideal de magnésio no solo}                       \\
        \hline
        Sumário        & O sistema deve permitir o usuário alterar os dados sobre a correção do solo. \\
        \hline
        Pré-condições  & \begin{itemize}
            \item O usuário deve estar logado
            \item O usuário deverá acessar uma ??análise?? já existente
        \end{itemize}                                                   \\
        \hline
        Atores         & Usuário                                                                      \\
        \hline
        Descrição      &
        \begin{itemize}
            \item Informa o percentual de cálcio desejado na CTC
            \item Seleciona qual fonte de cálcio será utilizada na correção
            \item Informa o custo médio em R\$/ha do corretivo
            \item Informa o percentual de PRNT do corretivo
            \item Informa o teor de CaO do corretivo
        \end{itemize}                                                                    \\
        \hline
        Alternativas   &
        \begin{itemize}
            \item Caso o usuário não saiba o teor de CaO, será utilizado o valor médio do corretivo
        \end{itemize}                                                                    \\
        \hline
        Exceção        &
        \begin{itemize}
            \item Caso o usuário informe um valor percental fora do intervalo de 0 a 100, não poderá proseguir com o preenchimento de outros campos.
            \item Caso o usuário informe um corretivo que não exista, uma mensagem de erro deverá ser exibida.
        \end{itemize}                                                                    \\
        \hline
    \end{tabular}
\end{quadro}

\begin{quadro}[H]
    \begin{tabular}{|p{3cm}|p{11cm}|}
        \hline
        \textbf{RS310} & \textbf{Calcular e exibir o valor do magnésio no solo após as correções}     \\
        \hline
        Sumário        & O sistema deve permitir o usuário alterar os dados sobre a correção do solo. \\
        \hline
        Pré-condições  & \begin{itemize}
            \item O usuário deve estar logado
            \item O usuário deverá acessar uma ??análise?? já existente
        \end{itemize}                                                   \\
        \hline
        Atores         & Usuário                                                                      \\
        \hline
        Descrição      &
        \begin{itemize}
            \item Informa o percentual de cálcio desejado na CTC
            \item Seleciona qual fonte de cálcio será utilizada na correção
            \item Informa o custo médio em R\$/ha do corretivo
            \item Informa o percentual de PRNT do corretivo
            \item Informa o teor de CaO do corretivo
        \end{itemize}                                                                    \\
        \hline
        Alternativas   &
        \begin{itemize}
            \item Caso o usuário não saiba o teor de CaO, será utilizado o valor médio do corretivo
        \end{itemize}                                                                    \\
        \hline
        Exceção        &
        \begin{itemize}
            \item Caso o usuário informe um valor percental fora do intervalo de 0 a 100, não poderá proseguir com o preenchimento de outros campos.
            \item Caso o usuário informe um corretivo que não exista, uma mensagem de erro deverá ser exibida.
        \end{itemize}                                                                    \\
        \hline
    \end{tabular}
\end{quadro}

\begin{quadro}[H]
    \begin{tabular}{|p{3cm}|p{11cm}|}
        \hline
        \textbf{RS312} & \textbf{Calcular e exibir a quantidade de corretivo a ser aplicada no solo.}                           \\
        \hline
        Sumário        & O sistema deve calcular a quantidade a ser aplicada, em toneladas por hectare, do corretivo informado. \\
        \hline
        Pré-condições  & \begin{itemize}
            \item O usuário deve estar logado
            \item O usuário deverá acessar uma ??análise?? já existente
        \end{itemize}                                                                             \\
        \hline
        Atores         & Usuário                                                                                                \\
        \hline
        Descrição      &
        \begin{itemize}
            \item Informa o percentual de cálcio desejado na CTC
            \item Seleciona qual fonte de cálcio será utilizada na correção
            \item Informa o custo médio em R\$/ha do corretivo
            \item Informa o percentual de PRNT do corretivo
            \item Informa o teor de CaO do corretivo
        \end{itemize}                                                                                              \\
        \hline
        Alternativas   &
        \begin{itemize}
            \item Caso o usuário não saiba o teor de CaO, será utilizado o valor médio do corretivo
        \end{itemize}                                                                                              \\
        \hline
        Exceção        &
        \begin{itemize}
            \item Caso o usuário informe um valor percental fora do intervalo de 0 a 100, não poderá proseguir com o preenchimento de outros campos.
            \item Caso o usuário informe um corretivo que não exista, uma mensagem de erro deverá ser exibida.
        \end{itemize}                                                                                              \\
        \hline
    \end{tabular}
\end{quadro}

\begin{quadro}[H]
    \begin{tabular}{|p{3cm}|p{11cm}|}
        \hline
        \textbf{RS314} & \textbf{Exibir a quantidade de enxofre que o corretivo fornecerá à terra}                                      \\
        \hline
        Sumário        & O sistema deve exibir o valor, em quilogramas por hectare, da quantidade de enxofre que o corretivo fornecerá. \\
        \hline
        Pré-condições  & \begin{itemize}
            \item O usuário deve estar logado
            \item O usuário deverá acessar uma ??análise?? já existente
        \end{itemize}                                                                                     \\
        \hline
        Atores         & Usuário                                                                                                        \\
        \hline
        Descrição      &
        \begin{itemize}
            \item Informa o percentual de cálcio desejado na CTC
            \item Seleciona qual fonte de cálcio será utilizada na correção
            \item Informa o custo médio em R\$/ha do corretivo
            \item Informa o percentual de PRNT do corretivo
            \item Informa o teor de CaO do corretivo
        \end{itemize}                                                                                                      \\
        \hline
        Alternativas   &
        \begin{itemize}
            \item Caso o usuário não saiba o teor de CaO, será utilizado o valor médio do corretivo
        \end{itemize}                                                                                                      \\
        \hline
        Exceção        &
        \begin{itemize}
            \item Caso o usuário informe um valor percental fora do intervalo de 0 a 100, não poderá proseguir com o preenchimento de outros campos.
            \item Caso o usuário informe um corretivo que não exista, uma mensagem de erro deverá ser exibida.
        \end{itemize}                                                                                                      \\
        \hline
    \end{tabular}
\end{quadro}

\begin{quadro}[H]
    \begin{tabular}{|p{3cm}|p{11cm}|}
        \hline
        \textbf{RS315} & \textbf{Exibir a quantidade de enxofre necessária}                                                              \\
        \hline
        Sumário        & O sistema deve exibir a quantidade suficiente, em quilogramas por hectare, da quantidade de enxofre necessária. \\
        \hline
        Pré-condições  & \begin{itemize}
            \item O usuário deve estar logado
            \item O usuário deverá acessar uma ??análise?? já existente
        \end{itemize}                                                                                      \\
        \hline
        Atores         & Usuário                                                                                                         \\
        \hline
        Descrição      &
        \begin{itemize}
            \item Informa o percentual de cálcio desejado na CTC
            \item Seleciona qual fonte de cálcio será utilizada na correção
            \item Informa o custo médio em R\$/ha do corretivo
            \item Informa o percentual de PRNT do corretivo
            \item Informa o teor de CaO do corretivo
        \end{itemize}                                                                                                       \\
        \hline
        Alternativas   &
        \begin{itemize}
            \item Caso o usuário não saiba o teor de CaO, será utilizado o valor médio do corretivo
        \end{itemize}                                                                                                       \\
        \hline
        Exceção        &
        \begin{itemize}
            \item Caso o usuário informe um valor percental fora do intervalo de 0 a 100, não poderá proseguir com o preenchimento de outros campos.
            \item Caso o usuário informe um corretivo que não exista, uma mensagem de erro deverá ser exibida.
        \end{itemize}                                                                                                       \\
        \hline
    \end{tabular}
\end{quadro}

\begin{quadro}[H]
    \begin{tabular}{|p{3cm}|p{11cm}|}
        \hline
        \textbf{RS316} & \textbf{Calcular e exibir o valor de V\% atual, ideal e após as correções}           \\
        \hline
        Sumário        & O sistema deve calcular e exibir os valores de V\% atual, ideal e após as correções. \\
        \hline
        Pré-condições  & \begin{itemize}
            \item O usuário deve estar logado
            \item A etapa de preenchimento dos dados da propriedade deverá estar preenchida
            \item A etapa de preenchimento da análise do solo deverá estar preenchida
            \item A etapa de preenchimento da matéria orgânica deverá está preenchida
            \item A etapa de preenchimento da correção do fósforo deverá estar preenchida
            \item A etapa de preenchimento da correção do postássio deverá estar preenchida
            \item A etapa de preenchimento da correção do cálcio e magnésio' deverá estar preenchida
        \end{itemize}                                                           \\
        \hline
        Atores         & Usuário                                                                              \\
        \hline
        Descrição      &
        \begin{itemize}
            \item O sistema realiza o cálculo do V\% atual
            \item O sistema exibe o resultado do cálculo do V\% atual
            \item O sistema realiza o cálculo do V\% após as correções
            \item O sistema exibe o resultado do cálculo do V\% após as correções
            \item O sistema realiza o cálculo do V\% ideal
            \item O sistema exibe o resultado do cálculo do V\% ideal
        \end{itemize}                                                                            \\
        \hline
        Alternativas   &
        \begin{itemize}
            \item Sem alternativas
        \end{itemize}                                                                            \\
        \hline
        Exceção        &
        \begin{itemize}
            \item Sem exceções
        \end{itemize}                                                                            \\
        \hline
    \end{tabular}
\end{quadro}