% INTRODUÇÃO-------------------------------------------------------------------

\chapter{INTRODUÇÃO}
\label{chap:introducao}

% A agricultura é um dos grandes pilares que sustentam a economia brasileira. Na safra 2018/2019, o Valor Bruto da Produção (VBP) referente ao setor foi de R\$ 372 bilhões \cite{minagri2019}. O Estado do Paraná, mesmo representando apenas 2\% do território \cite{ibge2018}, foi responsável por cerca de 19\% desse valor, além de empregar direta ou indiretamente 1 milhão de pessoas no estado \cite{emater2016}. A crescente produtividade é determinante para a redução do preço da alimentação, contribuindo para a melhora na qualidade de vida e saúde da população, fornecendo subsídios para o consumo de bens produzidos pela indústria e o setor de serviços \cite{cna2018}.

A produtividade das lavouras está intimamente ligada à qualidade do solo, que é responsável por levar a água e nutrientes aos cultivos e reciclar a matéria orgânica. Em áreas com um equilíbrio de nutrientes necessários para o desenvolvimento vegetal, a produtividade tende a ser maior. Devido a esse fato, faz-se necessário que os agricultores tomem decisões com maior agilidade para evitar a perda de rendimento.

% Para agilizar o processo de análise e correção do solo, a Empresa de Assistência Técnica e Extensão Rural (Emater), por meio de seus colaboradores Pedro Cecere Filho e Fabianderson José Baio, desenvolveu uma planilha de cálculo para o equilíbrio e correção do solo, que será apresentada na \autoref{subsec:planilha} deste documento.

Para auxiliar no processo de análise e correção do solo, a Empresa de Assistência Técnica e Extensão Rural (Emater), por meio de seus colaboradores Pedro Cecere Filho e Fabianderson José Baio, desenvolveu uma planilha de cálculo para o equilíbrio e correção do solo, que será apresentada na \autoref{subsec:planilha} deste documento.

\section{JUSTIFICATIVA}
\label{sec:justificativa}

A cada nova versão da planilha eletrônica, esta deve ser compartilhada entre os técnicos agrícolas, para que estes façam o uso das últimas funcionalidades implementadas. Devido a esse fato, apesar do uso de planilhas eletrônicas satisfazerem em partes as necessidades dos técnicos, não é a melhor forma para trabalhar com essas informações.

\begin{itemize}
    
    \item \textit{\textbf{Segurança}}: As planilhas não possuem controle de versão, por isso não é possível identificar quem inseriu um trecho malicioso ou uma informação errada no documento.

    \item \textit{\textbf{Integridade}}: Comumente os dados preenchidos na planilha são persistidos no próprio arquivo, devido a isso, existe a chance da corrupção da planilha e o usuário perder as informações inseridas.
    
    \item \textit{\textbf{Histórico}}: É muito difícil, ou mesmo impossível, gerenciar o histórico de mudanças em uma planilha. 
    
    \item \textit{\textbf{Armazenamento}}: Em alguns casos, o usuário duplica a planilha para executar uma nova operação. Essa duplicação gera o desperdício de disco, visto que, comumente, as diferenças entre planilhas são apenas nas células de entrada de dados. 
    
    \item \textit{\textbf{Versionamento}}: Quando uma nova versão de uma planilha surge, é muito trabalhoso a transposição de dados, favorecendo o surgimento de falhas.
    
    \item \textit{\textbf{Dependência de Software}}: A planilha depende do Microsoft Excel para funcionar adequadamente, o que acaba tornando os técnicos dependentes dos sistemas operacionais para os quais esse \textit{software} é desenvolvido.
    
    \item \textit{\textbf{Integração}}: É muito difícil integrar uma planilha com serviços disponíveis em rede. Por exemplo, é muito custoso integrar uma planilha com um \textit{webservice} que informa a cotação de moedas.
    
    \item \textit{\textbf{Padronização}}: Planilhas permitem alteração de cores, de fontes, movimentação de linhas e colunas, favorecendo o surgimento de falhas.
    
\end{itemize}

Devido a essas necessidades não supridas pela planilha, surgiu a necessidade do desenvolvimento de uma implementação da planilha como uma aplicação personalizada.

\section{OBJETIVOS}
\label{sec:objetivos}

% O objetivo deste documento é propor uma aplicação web responsiva para implementar as funcionalidades da planilha, organizando o processo de correção do solo a fim de solucionar os problemas de versionamento, independência de sistema operacional e do Microsoft Excel, armazenamento de dados e segurança.

O objetivo deste trabalho desenvolver uma aplicação web responsiva para implementar as funcionalidades atualmente disponíveis na planilha.

Os objetivos específicos deste trabalho são:

\begin{itemize}
    \item Desenvolver uma interface responsiva para a correção de nutrientes do solo
    
    \item Implementar as funcionalidades da pasta de trabalho descritas na  \autoref{subsec:pastaequilibriocorrecaoctc}.
    
    % \item Utilizar os conceitos de layouts responsivos, adequando a disposição dos elementos de acordo com a tela do dispositivo.
    
    % \item Proporcionar consultas à correções do solo feitas a partir da implantação da aplicação.
    
    \item Disponibilizar a aplicação na web como um serviço
\end{itemize}

\section{ORGANIZAÇÃO DO TRABALHO}
\label{sec:organizacaoTrabalho}

O restante desta proposta está organizada desta forma. O \autoref{chap:fundamentacao} faz uma abordagem ao referencial teórico que fundamentou o trabalho, bem como a apresentação da planilha que deu origem ao trabalho. Logo após, no \autoref{chap:proposta}, serão apresentadas as tecnologias e ferramentas que serão utilizadas no desenvolvimento da aplicação, além da metodologia e arquitetura que serão aplicadas no projeto. Ainda nesse capítulo serão apresentados os requisitos do sistema e os diagramas que modelam o aplicativo. Por fim, será apresentada a conclusão da da presente proposta no \autoref{chap:conclusao}.
