% REQUISITOS------------------------------------------------------------------

\section{ANÁLISE E PROJETO}
\label{chap:analise}
Esta seção apresenta os requisitos funcionais e não funcionais, a especificação dos requisitos funcionais, e os diagramas de caso de uso, classes e entidade-relacionamento.

\section{Requisitos Funcionais}
\label{sec:titSecReqFunc}

Texto...

\begin{longtable}{|p{1.5cm}|p{3cm}|p{7cm}|p{2.5cm}|}
    \hline
    ID    & REQUISITO                                                                        & DESCRIÇÃO                                                                                                                                                                                                                                                                                               & PRIORIDADE \\\hline
    \endfirsthead
    %
    \hline
    ID    & REQUISITO                                                                        & DESCRIÇÃO                                                                                                                                                                                                                                                                                               & PRIORIDADE \\\hline
    \endhead
    %
    RF001 & Gerenciar informações da propriedade.                                            & O sistema deve permitir ao usuário cadastrar, remover, alterar ou listar informações do solo de um produtor.                                                                                                                                                                                            & Essencial  \\\hline
    RF002 & Gerenciar nutrientes do solo.                                                    & O sistema deve permitir ao usuário inserir, remover, alterar ou listar os nutrientes coletados na análise do solo.                                                                                                                                                                                      & Essencial  \\\hline
    RF003 & Exibir valores ideais.                                                           & O sistema deve exibir os valores ideais de cada teor inserido de acordo com a textura do solo.                                                                                                                                                                                                          & Importante \\\hline
    RF004 & Exibir teores após correções.                                                    & O sistema deve exibir os valores dos teores inseridos após as correções.                                                                                                                                                                                                                                & Importante \\\hline
    RF005 & Gerenciar matéria orgânica.                                                      & O sistema deve permitir ao usuário inserir, remover, alterar ou listar o teor da matéria orgânica (M.O.) do solo.                                                                                                                                                                                       & Essencial  \\\hline
    RF006 & Exibir teor ideal da M.O.                                                        & O sistema deve exibir o teor ideal de matéria orgânica (M.O.).                                                                                                                                                                                                                                          & Importante \\\hline
    RF007 & Gerenciar dados relacionados à correção/recuperação  de Fósforo.                 & O sistema deve permitir ao usuário cadastrar, remover,  alterar ou listar valores ligados ao Fósforo, tais como teor a ser atingido, fonte a ser utilizada e eficiência.                                                                                                                                & Essencial  \\\hline
    RF008 & Exibir valores a serem aplicados no processo de correção/recuperação de Fósforo. & O sistema deve informar ao usuário a quantidade de Fósforo à ser aplicada assim como o seu custo.                                                                                                                                                                                                       & Importante \\\hline
    RF009 & Gerenciar valor/ton. (R\$) no processo relacionado às fontes de Fósforo.         & O sistema deve possibilitar ao usuário inserir, remover, alterar ou listar valores relacionados às fontes: superfosfato simples, superfosfato triplo, MAP, DAP, Yoorin, Fosfato Arad, Fosfato Gafsa, Fosfato Daoui, Fosfato  Patos Minas, Escória de Thomas, ácido Fosfórico e multifosfato magnesiano. & Essencial  \\\hline
    RF010 & Gerenciar dados relacionados à correção/recuperação do Potássio.                 & O sistema deve possibilitar ao usuário inserir, remover,  alterar ou listar dados relacionados à participação do Potássio na CTC assim como a fonte de Potássio a ser usada.                                                                                                                            & Essencial  \\\hline
    RF011 & Exibir valores relacionados à correção/recuperação do Potássio.                  & O sistema deve informar ao usuário dados à respeito do Potássio tais como: participação atual na CTC do solo,  participação do Potássio na CTC após correção, quantidade e custo a ser aplicada e sua participação ideal na CTC.                                                                        & Importante \\\hline
    RF012 & Gerenciar valor/ton. (R\$) no processo relacionado às fontes de Potássio.        & O sistema deve permitir ao usuário cadastrar, excluir, alterar ou listar valores como: cloreto de Potássio, Sulfato de Potássio e Sulfato de Potássio/Magnésio.                                                                                                                                         & Essencial  \\\hline
    RF300 & Cadastrar dados para a correção do cálcio e magnésio no solo                     & O sistema deve permitir o usuário informar dados acerca da correção do cálcio no solo.                                                                                                                                                                                                                  & Importante \\\hline
    RF301 & Alterar dados para a correção do cálcio e magnésio no solo                       & O sistema deve permitir o usuário alterar os dados sobre a correção do solo.                                                                                                                                                                                                                            & Essencial  \\\hline
    RF306 & Exibir o teor de cálcio atualmente no solo                                       & O sistema deve apresentar ao usuário, o percentual de Cálcio na CTC atual (antes das correções)                                                                                                                                                                                                         & Essencial  \\\hline
    RF307 & Exibir o teor de cálcio ideal no solo                                            & O sistema deve apresentar ao usuário, o intervalo percentual de Cálcio ideal na CTC.                                                                                                                                                                                                                    & Essencial  \\\hline
    RF308 & Calcular o percentual de cálcio ideal no solo                                    & O sistema deve apresentar ao usuário, o percentual ideal de Cálcio na CTC.                                                                                                                                                                                                                              & Essencial  \\\hline
    RF309 & Exibir teor de magnésio atualmente no solo                                       & O sistema deve apresentar ao usuário o percentual de Magnésio atual na CTC.                                                                                                                                                                                                                             & Essencial  \\\hline
    RF310 & Exibir o percentual ideal de magnésio no solo                                    & O sistema deve apresentar ao usuário o percentual de Magnésio ideal na CTC.                                                                                                                                                                                                                             & Essencial  \\\hline
    RF311 & Exibir o valor do magnésio no solo após as correções                             & O sistema deve apresentar ao usuário o percentual de Magnésio na CTC após as correções.                                                                                                                                                                                                                 & Essencial  \\\hline
    RF312 & Calcular o teor de magnésio no solo após as correções                            & O sistema deve calcular o percentual de Magnésio na CTC após as correções.                                                                                                                                                                                                                              & Essencial  \\\hline
    RF313 & Exibir a quantidade de corretivo de cálcio e magnésio a ser aplicada no solo.    & O sistema deve exibir a quantidade a ser aplicada, em toneladas por hectare, do corretivo informado.                                                                                                                                                                                                    & Essencial  \\\hline
    RF314 & Calcular a quantidade de corretivo a ser aplicada no solo.                       & O sistema deve calcular a quantidade a ser aplicada, em toneladas por hectare, do corretivo informado.                                                                                                                                                                                                  & Essencial  \\\hline
    RF315 & Exibir a quantidade de enxofre que o corretivo fornecerá à terra                 & O sistema deve exibir o valor, em quilogramas por hectare, da quantidade de enxofre que o corretivo fornecerá.                                                                                                                                                                                          & Desejável  \\\hline
    RF316 & Exibir a quantidade de enxofre necessária                                        & O sistema deve exibir a quantidade suficiente, em quilogramas por hectare, da quantidade de enxofre necessária.                                                                                                                                                                                         & Desejável  \\\hline
    RF317 & Calcular e exibir o valor de V\% atual, ideal e após as correções                & O sistema deve calcular e exibir os valores de V\% atual, ideal e após as correções.                                                                                                                                                                                                                    & Essencial  \\\hline\hline
\end{longtable}

\clearpage

\subsection{Especificação de Requisitos}
\label{sec:titSecEspReq}

Texto...
\begingroup

\begin{enumerate}
\def\labelenumi{\arabic{enumi}.}
\setcounter{enumi}{1}
\itemsep1pt\parskip0pt\parsep0pt
\item
  ESPECIFICAÇÃO DE REQUISITOS DO SISTEMA
\\\end{enumerate}

\begin{longtable}[c]{@{}|p{4cm}|p{9cm}|@{}}
\hline
\begin{minipage}[t]{0.47\columnwidth}
\textbf{RS001}
\end{minipage} & \begin{minipage}[t]{0.47\columnwidth}
Cadastrar informações do solo
\end{minipage}
\\\hline
\begin{minipage}[t]{0.47\columnwidth}
SUMÁRIO
\end{minipage} & \begin{minipage}[t]{0.47\columnwidth}
O requisito é responsável por cadastrar informações do solo de um
produtor.
\end{minipage}
\\\hline
\begin{minipage}[t]{0.47\columnwidth}
PRÉ-CONDIÇÕES
\end{minipage} & \begin{minipage}[t]{0.47\columnwidth}
\end{minipage}
\\\hline
\begin{minipage}[t]{0.47\columnwidth}
ATORES
\end{minipage} & \begin{minipage}[t]{0.47\columnwidth}
Usuário.
\end{minipage}
\\\hline
\begin{minipage}[t]{0.47\columnwidth}
DESCRIÇÃO
\end{minipage} & \begin{minipage}[t]{0.47\columnwidth}
\begin{enumerate}
\def\labelenumi{\arabic{enumi}.}
\itemsep1pt\parskip0pt\parsep0pt
\item
  O usuário loga no sistema.
\item
  O sistema exibe uma tela com botões referentes ao gerenciamento de uma
  propriedade.
\item
  O usuário clica sobre o botão ``Cadastrar Nova Propriedade''.
\item
  O usuário cadastra as informações do solo de um produtor que são
  exibidas na forma de um formulário, com os seguintes atributos: nome do produtor, data, município, lote, área total, talhão, área do talhão, matrícula do lote, textura de solo (argiloso ou textura média), sistema de cultivo (plantio direto ou convencional), profundidade da amostra de solos e número do resultado da análise de solos.
\def\labelenumi{\arabic{enumi}.}
\setcounter{enumi}{4}
\itemsep1pt\parskip0pt\parsep0pt
\item
  O usuário clica sobre o botão continuar ou salvar.
\item
  O sistema registra o cadastro da propriedade.
\\\end{enumerate}
\end{minipage}
\\\hline
\begin{minipage}[t]{0.47\columnwidth}
ALTERNATIVAS
\end{minipage} & \begin{minipage}[t]{0.47\columnwidth}
\begin{enumerate}
\def\labelenumi{\arabic{enumi}.}
\itemsep1pt\parskip0pt\parsep0pt
\item
  Caso não possua o nome do produtor no dropdown, o sistema permitirá
  inserir um novo nome.
\item
  Caso não possua o nome do município no dropdown, o sistema permitirá
  inserir um novo município.
\\\end{enumerate}
\end{minipage}
\\\hline
\begin{minipage}[t]{0.47\columnwidth}
EXCEÇÃO
\end{minipage} & \begin{minipage}[t]{0.47\columnwidth}
\begin{enumerate}
\def\labelenumi{\arabic{enumi}.}
\itemsep1pt\parskip0pt\parsep0pt
\item
  Caso o nome do produtor já exista, uma mensagem informando que este já
  está cadastrado no sistema será exibida.
\\\end{enumerate}
\end{minipage}
\\\hline

\end{longtable}

\begin{longtable}[c]{@{}|p{4cm}|p{9cm}|@{}}
\hline
\begin{minipage}[t]{0.47\columnwidth}
\textbf{RS002}
\end{minipage} & \begin{minipage}[t]{0.47\columnwidth}
Remover cadastro de propriedade
\end{minipage}
\\\hline
\begin{minipage}[t]{0.47\columnwidth}
SUMÁRIO
\end{minipage} & \begin{minipage}[t]{0.47\columnwidth}
O requisito é responsável por remover um cadastro com informações do
solo de um produtor.
\end{minipage}
\\\hline
\begin{minipage}[t]{0.47\columnwidth}
PRÉ-CONDIÇÕES
\end{minipage} & \begin{minipage}[t]{0.47\columnwidth}
O usuário deve ter realizado o cadastro de propriedade de algum produtor
referente ao requisito de sistema \textbf{RS001}.
\end{minipage}
\\\hline
\begin{minipage}[t]{0.47\columnwidth}
ATORES
\end{minipage} & \begin{minipage}[t]{0.47\columnwidth}
Usuário.
\end{minipage}
\\\hline
\begin{minipage}[t]{0.47\columnwidth}
DESCRIÇÃO
\end{minipage} & \begin{minipage}[t]{0.47\columnwidth}
\begin{enumerate}
\def\labelenumi{\arabic{enumi}.}
\itemsep1pt\parskip0pt\parsep0pt
\item
  O usuário loga no sistema.
\item
  O sistema exibe uma tela com botões na parte superior referentes ao
  gerenciamento de uma propriedade, e na parte inferior exibe os
  cadastros realizados pelo usuário na forma de uma tabela.
\item
  O usuário clica sobre um ícone em forma de ``X'' no canto direito do
  cadastro.
\item
  O sistema exibe uma mensagem de confirmação ao usuário.
\item
  O usuário clica em OK.
\item
  O sistema remove o cadastro referente.
\\\end{enumerate}
\end{minipage}
\\\hline
\begin{minipage}[t]{0.47\columnwidth}
ALTERNATIVAS
\end{minipage} & \begin{minipage}[t]{0.47\columnwidth}
O usuário também pode remover o cadastro da propriedade através do botão
de editar que é exibido na forma de um ícone na parte direita do
cadastro. Nesse caso, a remoção dos dados será manual.
\end{minipage}
\\\hline
\begin{minipage}[t]{0.47\columnwidth}
EXCEÇÃO
\end{minipage} & \begin{minipage}[t]{0.47\columnwidth}
\begin{enumerate}
\def\labelenumi{\arabic{enumi}.}
\itemsep1pt\parskip0pt\parsep0pt
\item
  No ato da exclusão, uma mensagem de aviso deverá ser exibida,
  informando ao usuário que todos os dados que dependem das informações
  do solo de um produtor serão também excluídas.
\\\end{enumerate}
\end{minipage}
\\\hline

\end{longtable}

\begin{longtable}[c]{@{}|p{4cm}|p{9cm}|@{}}
\hline
\begin{minipage}[t]{0.47\columnwidth}
\textbf{RS003}
\end{minipage} & \begin{minipage}[t]{0.47\columnwidth}
Alterar informações do solo
\end{minipage}
\\\hline
\begin{minipage}[t]{0.47\columnwidth}
SUMÁRIO
\end{minipage} & \begin{minipage}[t]{0.47\columnwidth}
O requisito é responsável por alterar informações do solo de um
produtor.
\end{minipage}
\\\hline
\begin{minipage}[t]{0.47\columnwidth}
PRÉ-CONDIÇÕES
\end{minipage} & \begin{minipage}[t]{0.47\columnwidth}
O usuário deve ter realizado o cadastro de propriedade de algum produtor
referente ao requisito de sistema \textbf{RS001}.
\end{minipage}
\\\hline
\begin{minipage}[t]{0.47\columnwidth}
ATORES
\end{minipage} & \begin{minipage}[t]{0.47\columnwidth}
Usuário.
\end{minipage}
\\\hline
\begin{minipage}[t]{0.47\columnwidth}
DESCRIÇÃO
\end{minipage} & \begin{minipage}[t]{0.47\columnwidth}
\begin{enumerate}
\def\labelenumi{\arabic{enumi}.}
\itemsep1pt\parskip0pt\parsep0pt
\item
  O usuário loga no sistema.
\item
  O sistema exibe uma tela com botões na parte superior referentes ao
  gerenciamento de uma propriedade, e na parte inferior exibe os
  cadastros realizados pelo usuário na forma de uma ``tabela''.
\item
  O usuário clica sobre o botão de editar que é exibido na forma de um
  ícone na parte direita do cadastro.
\item
  O mesmo formulário que é exibido no requisito de sistema \textbf{RS001} é
  exibido, com as informações do solo do produtor podendo ser alteradas.
\item
  O usuário clica sobre o botão continuar.
\item
  O sistema registra o cadastro da propriedade.
\\\end{enumerate}
\end{minipage}
\\\hline
\begin{minipage}[t]{0.47\columnwidth}
ALTERNATIVAS
\end{minipage} & \begin{minipage}[t]{0.47\columnwidth}
O usuário não precisa continuar o passo a passo das outras telas, basta
clicar sobre o botão salvar que automaticamente o sistema registra a
alteração.
\end{minipage}
\\\hline
\begin{minipage}[t]{0.47\columnwidth}
EXCEÇÃO
\end{minipage} & \begin{minipage}[t]{0.47\columnwidth}
\begin{enumerate}
\def\labelenumi{\arabic{enumi}.}
\itemsep1pt\parskip0pt\parsep0pt
\item
  Caso o nome do produtor já exista, uma mensagem informando que este já
  está cadastrado no sistema será exibida.
\item
  No ato da alteração, uma mensagem de aviso deverá ser exibida,
  informando ao usuário que todos os dados que dependem das informações
  do solo de um produtor serão também alterados.
\\\end{enumerate}
\end{minipage}
\\\hline

\end{longtable}

\begin{longtable}[c]{@{}|p{4cm}|p{9cm}|@{}}
\hline
\begin{minipage}[t]{0.47\columnwidth}
\textbf{RS004}
\end{minipage} & \begin{minipage}[t]{0.47\columnwidth}
Listar informações do solo
\end{minipage}
\\\hline
\begin{minipage}[t]{0.47\columnwidth}
SUMÁRIO
\end{minipage} & \begin{minipage}[t]{0.47\columnwidth}
O requisito é responsável por listar as informações do solo de um
produtor.
\end{minipage}
\\\hline
\begin{minipage}[t]{0.47\columnwidth}
PRÉ-CONDIÇÕES
\end{minipage} & \begin{minipage}[t]{0.47\columnwidth}
O usuário deve ter realizado o cadastro de propriedade de algum produtor
referente ao requisito de sistema \textbf{RS001}.
\end{minipage}
\\\hline
\begin{minipage}[t]{0.47\columnwidth}
ATORES
\end{minipage} & \begin{minipage}[t]{0.47\columnwidth}
Usuário.
\end{minipage}
\\\hline
\begin{minipage}[t]{0.47\columnwidth}
DESCRIÇÃO
\end{minipage} & \begin{minipage}[t]{0.47\columnwidth}
\begin{enumerate}
\def\labelenumi{\arabic{enumi}.}
\itemsep1pt\parskip0pt\parsep0pt
\item
  O usuário loga no sistema.
\item
  O sistema exibe uma tela com botões na parte superior referentes ao
  gerenciamento de uma propriedade, e na parte inferior exibe os
  cadastros realizados pelo usuário na forma de uma tabela.
\item
  O usuário clica sobre o cadastro que deseja listar as informações.
\item
  O sistema mostra para o usuário as informações referentes ao cadastro
  selecionado.
\\\end{enumerate}
\end{minipage}
\\\hline
\begin{minipage}[t]{0.47\columnwidth}
ALTERNATIVAS
\end{minipage} & \begin{minipage}[t]{0.47\columnwidth}
O usuário pode fazer a utilização do filtro na parte superior da tabela
através do nome do produtor.
\end{minipage}
\\\hline
\begin{minipage}[t]{0.47\columnwidth}
EXCEÇÃO
\end{minipage} & \begin{minipage}[t]{0.47\columnwidth}
\end{minipage}
\\\hline

\end{longtable}

\begin{longtable}[c]{@{}|p{4cm}|p{9cm}|@{}}
\hline
\begin{minipage}[t]{0.47\columnwidth}
\textbf{RS005}
\end{minipage} & \begin{minipage}[t]{0.47\columnwidth}
Cadastrar nutrientes do solo
\end{minipage}
\\\hline
\begin{minipage}[t]{0.47\columnwidth}
SUMÁRIO
\end{minipage} & \begin{minipage}[t]{0.47\columnwidth}
O requisito é responsável pela inserção dos nutrientes coletados na
análise do solo.
\end{minipage}
\\\hline
\begin{minipage}[t]{0.47\columnwidth}
PRÉ-CONDIÇÕES
\end{minipage} & \begin{minipage}[t]{0.47\columnwidth}
O usuário deve ter inserido as informações do solo do produtor
referentes ao requisito de sistema \textbf{RS001}.
\end{minipage}
\\\hline
\begin{minipage}[t]{0.47\columnwidth}
ATORES
\end{minipage} & \begin{minipage}[t]{0.47\columnwidth}
Usuário.
\end{minipage}
\\\hline
\begin{minipage}[t]{0.47\columnwidth}
DESCRIÇÃO
\end{minipage} & \begin{minipage}[t]{0.47\columnwidth}
\begin{enumerate}
\def\labelenumi{\arabic{enumi}.}
\itemsep1pt\parskip0pt\parsep0pt
\item
  O usuário loga no sistema.
\item
  O sistema exibe uma tela com botões na parte superior referentes ao
  gerenciamento de uma propriedade, e na parte inferior exibe os
  cadastros realizados pelo usuário na forma de uma tabela.
\item
  O usuário clica sobre o cadastro que deseja cadastrar os nutrientes do
  solo.
\item
  No formulário das informações da propriedade do produtor, o usuário
  clica no botão continuar.
\item
  O usuário informa os teores de Fósforo, Potássio, Cálcio Magnésio,
  Enxofre, Alumínio e H

  \begin{itemize}
  \itemsep1pt\parskip0pt\parsep0pt
  \item
    AL nos campos que são fornecidos na forma de um formulário. Os
    campos de Enxofre, Alumínio e H + AL não possuem obrigatoriedade de
    preenchimento.
  \end{itemize}
\item
  O usuário clica sobre o botão continuar.
\item
  O sistema registra a inserção dos teores.
\\\end{enumerate}
\end{minipage}
\\\hline
\begin{minipage}[t]{0.47\columnwidth}
ALTERNATIVAS
\end{minipage} & \begin{minipage}[t]{0.47\columnwidth}
\begin{enumerate}
\def\labelenumi{\arabic{enumi}.}
\itemsep1pt\parskip0pt\parsep0pt
\item
  O usuário também pode clicar sobre o botão salvar para que o cadastro
  dos nutrientes seja registrado.
\item
  O cadastro dos nutrientes do solo também pode ser feito após o
  cadastro das informações do solo referentes ao requisito do sistema
  \textbf{RS001}, através do botão continuar na parte inferior do formulário.
\\\end{enumerate}
\end{minipage}
\\\hline
\begin{minipage}[t]{0.47\columnwidth}
EXCEÇÃO
\end{minipage} & \begin{minipage}[t]{0.47\columnwidth}
\end{minipage}
\\\hline

\end{longtable}

\begin{longtable}[c]{@{}|p{4cm}|p{9cm}|@{}}
\hline
\begin{minipage}[t]{0.47\columnwidth}
\textbf{RS006}
\end{minipage} & \begin{minipage}[t]{0.47\columnwidth}
Remover nutrientes inseridos
\end{minipage}
\\\hline
\begin{minipage}[t]{0.47\columnwidth}
SUMÁRIO
\end{minipage} & \begin{minipage}[t]{0.47\columnwidth}
O requisito é responsável por remover um cadastro com informações do
solo de um produtor.
\end{minipage}
\\\hline
\begin{minipage}[t]{0.47\columnwidth}
PRÉ-CONDIÇÕES
\end{minipage} & \begin{minipage}[t]{0.47\columnwidth}
O usuário deve ter realizado a inserção dos nutrientes coletados na
análise do solo referente ao requisito de sistema \textbf{RS005}.
\end{minipage}
\\\hline
\begin{minipage}[t]{0.47\columnwidth}
ATORES
\end{minipage} & \begin{minipage}[t]{0.47\columnwidth}
Usuário.
\end{minipage}
\\\hline
\begin{minipage}[t]{0.47\columnwidth}
DESCRIÇÃO
\end{minipage} & \begin{minipage}[t]{0.47\columnwidth}
\begin{enumerate}
\def\labelenumi{\arabic{enumi}.}
\itemsep1pt\parskip0pt\parsep0pt
\item
  O usuário loga no sistema.
\item
  O sistema exibe uma tela com botões na parte superior referentes ao
  gerenciamento de uma propriedade, e na parte inferior exibe os
  cadastros realizados pelo usuário na forma de uma tabela.
\item
  O usuário clica sobre um ícone em forma de ``X'' no canto direito do
  cadastro.
\item
  O sistema exibe uma mensagem de confirmação ao usuário.
\item
  O usuário clica em OK.
\item
  O sistema remove o nutriente referente.
\\\end{enumerate}
\end{minipage}
\\\hline
\begin{minipage}[t]{0.47\columnwidth}
ALTERNATIVAS
\end{minipage} & \begin{minipage}[t]{0.47\columnwidth}
O usuário também pode remover os nutrientes inseridos através do botão
de editar que é exibido na forma de um ícone na parte direita do
cadastro. Nesse caso, a remoção dos dados será manual.
\end{minipage}
\\\hline
\begin{minipage}[t]{0.47\columnwidth}
EXCEÇÃO
\end{minipage} & \begin{minipage}[t]{0.47\columnwidth}
\begin{enumerate}
\def\labelenumi{\arabic{enumi}.}
\itemsep1pt\parskip0pt\parsep0pt
\item
  No ato da exclusão, uma mensagem de aviso deverá ser exibida,
  informando ao usuário que todos os dados que dependem das informações
  relacionadas aos nutrientes do solo serão excluídas.
\\\end{enumerate}
\end{minipage}
\\\hline

\end{longtable}

\begin{longtable}[c]{@{}|p{4cm}|p{9cm}|@{}}
\hline
\begin{minipage}[t]{0.47\columnwidth}
\textbf{RS008}
\end{minipage} & \begin{minipage}[t]{0.47\columnwidth}
Alterar nutrientes do solo
\end{minipage}
\\\hline
\begin{minipage}[t]{0.47\columnwidth}
SUMÁRIO
\end{minipage} & \begin{minipage}[t]{0.47\columnwidth}
O requisito é responsável por alterar nutrientes do solo que foram
inseridos.
\end{minipage}
\\\hline
\begin{minipage}[t]{0.47\columnwidth}
PRÉ-CONDIÇÕES
\end{minipage} & \begin{minipage}[t]{0.47\columnwidth}
O usuário deve ter realizado a inserção dos nutrientes coletados na
análise do solo referente ao requisito de sistema \textbf{RS005}.
\end{minipage}
\\\hline
\begin{minipage}[t]{0.47\columnwidth}
ATORES
\end{minipage} & \begin{minipage}[t]{0.47\columnwidth}
Usuário.
\end{minipage}
\\\hline
\begin{minipage}[t]{0.47\columnwidth}
DESCRIÇÃO
\end{minipage} & \begin{minipage}[t]{0.47\columnwidth}
\begin{enumerate}
\def\labelenumi{\arabic{enumi}.}
\itemsep1pt\parskip0pt\parsep0pt
\item
  O usuário loga no sistema.
\item
  O sistema exibe uma tela com botões na parte superior referentes ao
  gerenciamento de uma propriedade, e na parte inferior exibe os
  cadastros realizados pelo usuário na forma de uma tabela.
\item
  O usuário clica sobre o botão de editar que é exibido na forma de um
  ícone na parte direita do cadastro que deseja alterar.
\item
  No formulário das informações da propriedade do produtor, o usuário
  clica no botão continuar.
\item
  O usuário realiza as alterações nos teores do solo já inseridos.
\item
  O usuário clica no botão continuar.
\item
  O sistema registra a alteração feita.
\\\end{enumerate}
\end{minipage}
\\\hline
\begin{minipage}[t]{0.47\columnwidth}
ALTERNATIVAS
\end{minipage} & \begin{minipage}[t]{0.47\columnwidth}
O usuário também pode clicar sobre o botão salvar para que as alterações
dos nutrientes sejam registradas.
\end{minipage}
\\\hline
\begin{minipage}[t]{0.47\columnwidth}
EXCEÇÃO
\end{minipage} & \begin{minipage}[t]{0.47\columnwidth}
\begin{enumerate}
\def\labelenumi{\arabic{enumi}.}
\itemsep1pt\parskip0pt\parsep0pt
\item
  No ato da alteração, uma mensagem de aviso deverá ser exibida,
  informando ao usuário que todos os dados que dependem das informações
  relacionadas aos nutrientes do solo serão alteradas.
\\\end{enumerate}
\end{minipage}
\\\hline

\end{longtable}

\begin{longtable}[c]{@{}|p{4cm}|p{9cm}|@{}}
\hline
\begin{minipage}[t]{0.47\columnwidth}
\textbf{RS009}
\end{minipage} & \begin{minipage}[t]{0.47\columnwidth}
Listar nutrientes do solo
\end{minipage}
\\\hline
\begin{minipage}[t]{0.47\columnwidth}
SUMÁRIO
\end{minipage} & \begin{minipage}[t]{0.47\columnwidth}
O requisito é responsável por listar os nutrientes do solo de um
produtor que foram inseridos.
\end{minipage}
\\\hline
\begin{minipage}[t]{0.47\columnwidth}
PRÉ-CONDIÇÕES
\end{minipage} & \begin{minipage}[t]{0.47\columnwidth}
O usuário deve ter realizado a inserção dos nutrientes coletados na
análise do solo referente ao requisito de sistema \textbf{RS005}.
\end{minipage}
\\\hline
\begin{minipage}[t]{0.47\columnwidth}
ATORES
\end{minipage} & \begin{minipage}[t]{0.47\columnwidth}
Usuário.
\end{minipage}
\\\hline
\begin{minipage}[t]{0.47\columnwidth}
DESCRIÇÃO
\end{minipage} & \begin{minipage}[t]{0.47\columnwidth}
\begin{enumerate}
\def\labelenumi{\arabic{enumi}.}
\itemsep1pt\parskip0pt\parsep0pt
\item
  O usuário loga no sistema.
\item
  O sistema exibe uma tela com botões na parte superior referentes ao
  gerenciamento de uma propriedade, e na parte inferior exibe os
  cadastros realizados pelo usuário na forma de uma tabela.
\item
  O usuário clica sobre a inserção que deseja listar.
\item
  O sistema mostra para o usuário as informações referentes.
\\\end{enumerate}
\end{minipage}
\\\hline
\begin{minipage}[t]{0.47\columnwidth}
ALTERNATIVAS
\end{minipage} & \begin{minipage}[t]{0.47\columnwidth}
\end{minipage}
\\\hline
\begin{minipage}[t]{0.47\columnwidth}
EXCEÇÃO
\end{minipage} & \begin{minipage}[t]{0.47\columnwidth}
\end{minipage}
\\\hline

\end{longtable}

\begin{longtable}[c]{@{}|p{4cm}|p{9cm}|@{}}
\hline
\begin{minipage}[t]{0.47\columnwidth}
\textbf{RS010}
\end{minipage} & \begin{minipage}[t]{0.47\columnwidth}
Exibir valores ideais dos nutrientes inseridos
\end{minipage}
\\\hline
\begin{minipage}[t]{0.47\columnwidth}
SUMÁRIO
\end{minipage} & \begin{minipage}[t]{0.47\columnwidth}
O requisito é responsável por mostrar na tela os valores considerados
ideais para os nutrientes inseridos.
\end{minipage}
\\\hline
\begin{minipage}[t]{0.47\columnwidth}
PRÉ-CONDIÇÕES
\end{minipage} & \begin{minipage}[t]{0.47\columnwidth}
O usuário deve ter realizado a inserção dos nutrientes coletados na
análise do solo referente ao requisito de sistema \textbf{RS005}.
\end{minipage}
\\\hline
\begin{minipage}[t]{0.47\columnwidth}
ATORES
\end{minipage} & \begin{minipage}[t]{0.47\columnwidth}
Usuário.
\end{minipage}
\\\hline
\begin{minipage}[t]{0.47\columnwidth}
DESCRIÇÃO
\end{minipage} & \begin{minipage}[t]{0.47\columnwidth}
\begin{enumerate}
\def\labelenumi{\arabic{enumi}.}
\itemsep1pt\parskip0pt\parsep0pt
\item
  O usuário loga no sistema.
\item
  O sistema exibe uma tela com botões na parte superior referentes ao
  gerenciamento de uma propriedade, e na parte inferior exibe os
  cadastros realizados pelo usuário na forma de uma tabela.
\item
  O usuário clica sobre o cadastro que deseja ver os valores ideais dos
  nutrientes inseridos.
\item
  O sistema mostra para o usuário os valores ideais de cada nutriente
  inserido na parte inferior da tela, de acordo com a textura de solo
  feito no cadastro de informações do solo.
\item
  No caso de um solo ``Argiloso'', os seguintes valores serão exibidos:
  ``9,0'' para o Fósforo, ``0,35'' para o Potássio, ``6,0'' para o
  Cálcio, ``1,63'' para o Magnésio e ``3,67'' para o Enxofre.
\item
  Caso seja cadastrado um solo de ``Textura Média'', a exibição passa a
  ser: ``12,0'' para o Fósforo, ``0,25'' para o Potássio, ``4,0'' para o
  Cálcio, ``1,0'' para o Magnésio e ``6,0'' para o Enxofre.
\\\end{enumerate}
\end{minipage}
\\\hline
\begin{minipage}[t]{0.47\columnwidth}
ALTERNATIVAS
\end{minipage} & \begin{minipage}[t]{0.47\columnwidth}
A exibição dos valores considerados ideais dos nutrientes inseridos
também pode ser feito depois do cadastro das informações do solo
referentes ao requisito do sistema \textbf{RS001}, logo que o usuário insere os
nutrientes do solo.
\end{minipage}
\\\hline
\begin{minipage}[t]{0.47\columnwidth}
EXCEÇÃO
\end{minipage} & \begin{minipage}[t]{0.47\columnwidth}
\end{minipage}
\\\hline

\end{longtable}

\begin{longtable}[c]{@{}|p{4cm}|p{9cm}|@{}}
\hline
\begin{minipage}[t]{0.47\columnwidth}
\textbf{RS011}
\end{minipage} & \begin{minipage}[t]{0.47\columnwidth}
Exibir valores dos nutrientes após correções
\end{minipage}
\\\hline
\begin{minipage}[t]{0.47\columnwidth}
SUMÁRIO
\end{minipage} & \begin{minipage}[t]{0.47\columnwidth}
O requisito é responsável por mostrar na tela os valores corrigidos dos
nutrientes inseridos.
\end{minipage}
\\\hline
\begin{minipage}[t]{0.47\columnwidth}
PRÉ-CONDIÇÕES
\end{minipage} & \begin{minipage}[t]{0.47\columnwidth}
O usuário deve ter realizado a inserção dos nutrientes coletados na
análise do solo referente ao requisito de sistema \textbf{RS005}.
\end{minipage}
\\\hline
\begin{minipage}[t]{0.47\columnwidth}
ATORES
\end{minipage} & \begin{minipage}[t]{0.47\columnwidth}
Usuário.
\end{minipage}
\\\hline
\begin{minipage}[t]{0.47\columnwidth}
DESCRIÇÃO
\end{minipage} & \begin{minipage}[t]{0.47\columnwidth}
\begin{enumerate}
\def\labelenumi{\arabic{enumi}.}
\itemsep1pt\parskip0pt\parsep0pt
\item
  O usuário loga no sistema.
\item
  O sistema exibe uma tela com botões na parte superior referentes ao
  gerenciamento de uma propriedade, e na parte inferior exibe os
  cadastros realizados pelo usuário na forma de uma tabela.
\item
  O usuário clica sobre o cadastro que deseja ver os valores após as
  correções dos nutrientes inseridos.
\item
  O sistema exibe para cada nutriente o seu valor após as correções na
  parte inferior da tela.
\\\end{enumerate}
\end{minipage}
\\\hline
\begin{minipage}[t]{0.47\columnwidth}
ALTERNATIVAS
\end{minipage} & \begin{minipage}[t]{0.47\columnwidth}
\end{minipage}
\\\hline
\begin{minipage}[t]{0.47\columnwidth}
EXCEÇÃO
\end{minipage} & \begin{minipage}[t]{0.47\columnwidth}
\end{minipage}
\\\hline

\end{longtable}

\begin{longtable}[c]{@{}|p{4cm}|p{9cm}|@{}}
\hline
\begin{minipage}[t]{0.47\columnwidth}
\textbf{RS012}
\end{minipage} & \begin{minipage}[t]{0.47\columnwidth}
Cadastrar matéria orgânica
\end{minipage}
\\\hline
\begin{minipage}[t]{0.47\columnwidth}
SUMÁRIO
\end{minipage} & \begin{minipage}[t]{0.47\columnwidth}
O caso de uso é responsável pela inserção do teor da matéria orgânica do
solo.
\end{minipage}
\\\hline
\begin{minipage}[t]{0.47\columnwidth}
PRÉ-CONDIÇÕES
\end{minipage} & \begin{minipage}[t]{0.47\columnwidth}
\begin{enumerate}
\def\labelenumi{\arabic{enumi}.}
\itemsep1pt\parskip0pt\parsep0pt
\item
  O usuário deve ter inserido as informações do solo do produtor
  referentes ao requisito de sistema \textbf{RS001}.
\item
  O usuário deve ter inserido as informações dos teores coletados na
  análise do solo referentes ao requisito de sistema \textbf{RS005}.
\\\end{enumerate}
\end{minipage}
\\\hline
\begin{minipage}[t]{0.47\columnwidth}
ATORES
\end{minipage} & \begin{minipage}[t]{0.47\columnwidth}
Usuário.
\end{minipage}
\\\hline
\begin{minipage}[t]{0.47\columnwidth}
DESCRIÇÃO
\end{minipage} & \begin{minipage}[t]{0.47\columnwidth}
\begin{enumerate}
\def\labelenumi{\arabic{enumi}.}
\itemsep1pt\parskip0pt\parsep0pt
\item
  O usuário loga no sistema.
\item
  O sistema exibe uma tela com botões na parte superior referentes ao
  gerenciamento de uma propriedade, e na parte inferior exibe os
  cadastros realizados pelo usuário na forma de uma tabela.
\item
  O usuário clica sobre o cadastro que deseja cadastrar o a matéria
  orgânica do solo.
\item
  No formulário das informações da propriedade do produtor, o usuário
  clica no botão continuar.
\item
  O usuário informa a quantidade de matéria orgânica (g.dm³, porcentagem
  ou carbono).
\item
  O usuário clica sobre o botão continuar.
\item
  O sistema registra a inserção da matéria orgânica.
\\\end{enumerate}
\end{minipage}
\\\hline
\begin{minipage}[t]{0.47\columnwidth}
ALTERNATIVAS
\end{minipage} & \begin{minipage}[t]{0.47\columnwidth}
\begin{enumerate}
\def\labelenumi{\arabic{enumi}.}
\itemsep1pt\parskip0pt\parsep0pt
\item
  O usuário também pode clicar sobre o botão salvar para que o cadastro
  da matéria orgânica seja registrado.
\item
  O cadastro da matéria orgânica também pode ser feito após o cadastro
  das informações do solo referentes ao requisito do sistema \textbf{RS001},
  através do botão continuar na parte inferior do formulário.
\\\end{enumerate}
\end{minipage}
\\\hline
\begin{minipage}[t]{0.47\columnwidth}
EXCEÇÃO
\end{minipage} & \begin{minipage}[t]{0.47\columnwidth}
\end{minipage}
\\\hline

\end{longtable}

\begin{longtable}[c]{@{}|p{4cm}|p{9cm}|@{}}
\hline
\begin{minipage}[t]{0.47\columnwidth}
\textbf{RS013}
\end{minipage} & \begin{minipage}[t]{0.47\columnwidth}
Remover matéria orgânica inserida
\end{minipage}
\\\hline
\begin{minipage}[t]{0.47\columnwidth}
SUMÁRIO
\end{minipage} & \begin{minipage}[t]{0.47\columnwidth}
O requisito é responsável por remover as informações da matéria orgânica
inserida do solo de um produtor.
\end{minipage}
\\\hline
\begin{minipage}[t]{0.47\columnwidth}
PRÉ-CONDIÇÕES
\end{minipage} & \begin{minipage}[t]{0.47\columnwidth}
O usuário deve ter realizado a inserção da matéria orgânica do solo
referente ao requisito de sistema \textbf{RS012}.
\end{minipage}
\\\hline
\begin{minipage}[t]{0.47\columnwidth}
ATORES
\end{minipage} & \begin{minipage}[t]{0.47\columnwidth}
Usuário.
\end{minipage}
\\\hline
\begin{minipage}[t]{0.47\columnwidth}
DESCRIÇÃO
\end{minipage} & \begin{minipage}[t]{0.47\columnwidth}
\begin{enumerate}
\def\labelenumi{\arabic{enumi}.}
\itemsep1pt\parskip0pt\parsep0pt
\item
  O usuário loga no sistema.
\item
  O sistema exibe uma tela com botões na parte superior referentes ao
  gerenciamento de uma propriedade, e na parte inferior exibe os
  cadastros realizados pelo usuário na forma de uma tabela.
\item
  O usuário clica sobre um ícone em forma de ``X'' no canto direito do
  cadastro.
\item
  O sistema exibe uma mensagem de confirmação ao usuário.
\item
  O usuário clica em OK.
\item
  O sistema remove a matéria orgânica referente.
\\\end{enumerate}
\end{minipage}
\\\hline
\begin{minipage}[t]{0.47\columnwidth}
ALTERNATIVAS
\end{minipage} & \begin{minipage}[t]{0.47\columnwidth}
O usuário também pode remover a matéria orgânica inserida através do
botão de editar que é exibido na forma de um ícone na parte direita do
cadastro. Nesse caso, a remoção dos dados será manual.
\end{minipage}
\\\hline
\begin{minipage}[t]{0.47\columnwidth}
EXCEÇÃO
\end{minipage} & \begin{minipage}[t]{0.47\columnwidth}
\begin{enumerate}
\def\labelenumi{\arabic{enumi}.}
\itemsep1pt\parskip0pt\parsep0pt
\item
  No ato da exclusão, uma mensagem de aviso deverá ser exibida,
  informando ao usuário que todos os dados que dependem das informações
  relacionadas à matéria orgânica do solo também serão excluídas.
\\\end{enumerate}
\end{minipage}
\\\hline

\end{longtable}

\begin{longtable}[c]{@{}|p{4cm}|p{9cm}|@{}}
\hline
\begin{minipage}[t]{0.47\columnwidth}
\textbf{RS014}
\end{minipage} & \begin{minipage}[t]{0.47\columnwidth}
Alterar matéria orgânica do solo
\end{minipage}
\\\hline
\begin{minipage}[t]{0.47\columnwidth}
SUMÁRIO
\end{minipage} & \begin{minipage}[t]{0.47\columnwidth}
O requisito é responsável por alterar a matéria orgânica do solo que foi
inserida.
\end{minipage}
\\\hline
\begin{minipage}[t]{0.47\columnwidth}
PRÉ-CONDIÇÕES
\end{minipage} & \begin{minipage}[t]{0.47\columnwidth}
O usuário deve ter realizado a inserção da matéria orgânica coletados na
análise do solo referente ao requisito de sistema \textbf{RS012}.
\end{minipage}
\\\hline
\begin{minipage}[t]{0.47\columnwidth}
ATORES
\end{minipage} & \begin{minipage}[t]{0.47\columnwidth}
Usuário.
\end{minipage}
\\\hline
\begin{minipage}[t]{0.47\columnwidth}
DESCRIÇÃO
\end{minipage} & \begin{minipage}[t]{0.47\columnwidth}
\begin{enumerate}
\def\labelenumi{\arabic{enumi}.}
\item
  O usuário loga no sistema.
\item
  O sistema exibe uma tela com botões na parte superior referentes ao
  gerenciamento de uma propriedade, e na parte inferior exibe os
  cadastros realizados pelo usuário na forma de uma tabela.
\item
  O usuário clica sobre o botão de editar que é exibido na forma de um
  ícone na parte direita do cadastro que deseja alterar.
\item
  No formulário das informações da propriedade do produtor, o usuário
  clica no botão continuar.
\item
  O usuário realiza as alterações da matéria orgânica.
\item
  O usuário clica no botão continuar.
\item
  O sistema registra a alteração feita.
\\\end{enumerate}
\end{minipage}
\\\hline
\begin{minipage}[t]{0.47\columnwidth}
ALTERNATIVAS
\end{minipage} & \begin{minipage}[t]{0.47\columnwidth}
O usuário também pode clicar sobre o botão salvar para que a alteração
da matéria orgânica seja registrada.
\end{minipage}
\\\hline
\begin{minipage}[t]{0.47\columnwidth}
EXCEÇÃO
\end{minipage} & \begin{minipage}[t]{0.47\columnwidth}
\begin{enumerate}
\def\labelenumi{\arabic{enumi}.}
\itemsep1pt\parskip0pt\parsep0pt
\item
  No ato da alteração, uma mensagem de aviso deverá ser exibida,
  informando ao usuário que todos os dados que dependem das informações
  relacionadas à matéria orgânica do solo também serão alteradas.
\\\end{enumerate}
\end{minipage}
\\\hline

\end{longtable}

\begin{longtable}[c]{@{}|p{4cm}|p{9cm}|@{}}
\hline
\begin{minipage}[t]{0.47\columnwidth}
\textbf{RS015}
\end{minipage} & \begin{minipage}[t]{0.47\columnwidth}
Listar matéria orgânica
\end{minipage}
\\\hline
\begin{minipage}[t]{0.47\columnwidth}
SUMÁRIO
\end{minipage} & \begin{minipage}[t]{0.47\columnwidth}
O requisito é responsável por listar a matéria orgânica do solo de um
produtor que foi inserido.
\end{minipage}
\\\hline
\begin{minipage}[t]{0.47\columnwidth}
PRÉ-CONDIÇÕES
\end{minipage} & \begin{minipage}[t]{0.47\columnwidth}
O usuário deve ter realizado a inserção da matéria orgânica coletado na
análise do solo referente ao requisito de sistema \textbf{RS012}.
\end{minipage}
\\\hline
\begin{minipage}[t]{0.47\columnwidth}
ATORES
\end{minipage} & \begin{minipage}[t]{0.47\columnwidth}
Usuário.
\end{minipage}
\\\hline
\begin{minipage}[t]{0.47\columnwidth}
DESCRIÇÃO
\end{minipage} & \begin{minipage}[t]{0.47\columnwidth}
\begin{enumerate}
\def\labelenumi{\arabic{enumi}.}
\itemsep1pt\parskip0pt\parsep0pt
\item
  O usuário loga no sistema.
\item
  O sistema exibe uma tela com botões na parte superior referentes ao
  gerenciamento de uma propriedade, e na parte inferior exibe os
  cadastros realizados pelo usuário na forma de uma tabela.
\item
  O usuário clica sobre a inserção que deseja listar.
\item
  O sistema mostra para o usuário as informações referentes.
\\\end{enumerate}
\end{minipage}
\\\hline
\begin{minipage}[t]{0.47\columnwidth}
ALTERNATIVAS
\end{minipage} & \begin{minipage}[t]{0.47\columnwidth}
\end{minipage}
\\\hline
\begin{minipage}[t]{0.47\columnwidth}
EXCEÇÃO
\end{minipage} & \begin{minipage}[t]{0.47\columnwidth}
\end{minipage}
\\\hline

\end{longtable}

\begin{longtable}[c]{@{}|p{4cm}|p{9cm}|@{}}
\hline
\begin{minipage}[t]{0.47\columnwidth}
\textbf{RS017}
\end{minipage} & \begin{minipage}[t]{0.47\columnwidth}
Exibir teor ideal da Matéria Orgânica
\end{minipage}
\\\hline
\begin{minipage}[t]{0.47\columnwidth}
SUMÁRIO
\end{minipage} & \begin{minipage}[t]{0.47\columnwidth}
O requisito é responsável por mostrar na tela o valor considerado ideal
da matéria orgânica
\end{minipage}
\\\hline
\begin{minipage}[t]{0.47\columnwidth}
PRÉ-CONDIÇÕES
\end{minipage} & \begin{minipage}[t]{0.47\columnwidth}
O usuário deve ter realizado a inserção da matéria orgânica coletado na
análise do solo referente ao requisito de sistema \textbf{RS012}.
\end{minipage}
\\\hline
\begin{minipage}[t]{0.47\columnwidth}
ATORES
\end{minipage} & \begin{minipage}[t]{0.47\columnwidth}
Usuário.
\end{minipage}
\\\hline
\begin{minipage}[t]{0.47\columnwidth}
DESCRIÇÃO
\end{minipage} & \begin{minipage}[t]{0.47\columnwidth}
\begin{enumerate}
\def\labelenumi{\arabic{enumi}.}
\itemsep1pt\parskip0pt\parsep0pt
\item
  O usuário loga no sistema.
\item
  O sistema exibe uma tela com botões na parte superior referentes ao
  gerenciamento de uma propriedade, e na parte inferior exibe os
  cadastros realizados pelo usuário na forma de uma tabela.
\item
  O usuário clica sobre a inserção que deseja ver o teor ideal da
  matéria orgânica.
\item
  O sistema mostra para o usuário as informações referentes (o valor
  independe de qualquer variável, trata-se de um intervalo padrão de 3,0
  a 4,0\%).
\\\end{enumerate}
\end{minipage}
\\\hline
\begin{minipage}[t]{0.47\columnwidth}
ALTERNATIVAS
\end{minipage} & \begin{minipage}[t]{0.47\columnwidth}
A exibição do valor considerado ideal da matéria orgânica inserida
também pode ser feito depois do cadastro das informações do solo
referentes ao requisito do sistema \textbf{RS001}, logo que o usuário insere os
teores do solo e a matéria orgânica.
\end{minipage}
\\\hline
\begin{minipage}[t]{0.47\columnwidth}
EXCEÇÃO
\end{minipage} & \begin{minipage}[t]{0.47\columnwidth}
\end{minipage}
\\\hline

\end{longtable}

\begin{longtable}[c]{@{}|p{4cm}|p{9cm}|@{}}
\hline
\begin{minipage}[t]{0.47\columnwidth}
\textbf{RS018}
\end{minipage} & \begin{minipage}[t]{0.47\columnwidth}
Cadastrar informações relacionadas à correção/recuperação do Fósforo
\end{minipage}
\\\hline
\begin{minipage}[t]{0.47\columnwidth}
SUMÁRIO
\end{minipage} & \begin{minipage}[t]{0.47\columnwidth}
O requisito é responsável pela inserção de valores relacionados à
correção/recuperação do Fósforo.
\end{minipage}
\\\hline
\begin{minipage}[t]{0.47\columnwidth}
PRÉ-CONDIÇÕES
\end{minipage} & \begin{minipage}[t]{0.47\columnwidth}
O usuário deve ter inserido informações relacionadas ao Fósforo conforme
descrito no requisito de sistema \textbf{RS005}.
\end{minipage}
\\\hline
\begin{minipage}[t]{0.47\columnwidth}
ATORES
\end{minipage} & \begin{minipage}[t]{0.47\columnwidth}
Usuário
\end{minipage}
\\\hline
\begin{minipage}[t]{0.47\columnwidth}
DESCRIÇÃO
\end{minipage} & \begin{minipage}[t]{0.47\columnwidth}
\begin{enumerate}
\def\labelenumi{\arabic{enumi}.}
\itemsep1pt\parskip0pt\parsep0pt
\item
  O usuário loga no sistema;
\item
  O sistema exibe uma tela com botões referentes ao gerenciamento de
  informações relacionadas à correção/recuperação do Fósforo.
\item
  O usuário clica no botão ``Cadastrar correção/recuperação do
  Fósforo''.
\item
  O usuário insere os valores relacionados ao Fósforo tais como:
\\\end{enumerate}

\begin{itemize}
\itemsep1pt\parskip0pt\parsep0pt
\item
  teor a ser atingido;
\item
  fonte a ser utilizada;
\item
  percentual de eficiência.
\end{itemize}
\end{minipage}
\\\hline
\begin{minipage}[t]{0.47\columnwidth}
ALTERNATIVAS
\end{minipage} & \begin{minipage}[t]{0.47\columnwidth}
\begin{enumerate}
\def\labelenumi{\arabic{enumi}.}
\itemsep1pt\parskip0pt\parsep0pt
\item
  O campo referente à fonte a ser utilizada deve ser um valor inteiro
  preenchido em um intervalo de 0 à 12, sendo:
\\\end{enumerate}

\begin{itemize}
\itemsep1pt\parskip0pt\parsep0pt
\item
  ``1'' para Superfosfato Simples;
\item
  ``2'' para Superfosfato Triplo;
\item
  ``3'' para MAP; 
\item
  ``4'' para DAP;
\item
  ``5'' para Termofosfato Yoorin;
\item
  ``6'' para Fosfato Reativo Arad;
\item
  ``7'' para Fosfato Reativo de Gafsa;
\item
  ``8'' para Fosfato Reativo Daoui;
\item
  ``9'' para Fosfato Natural Patos de Minas;
\item
  ``10'' para Escória de Thomas;
\item
  ``11'' para Ácido Fosfórico;
\item
  ``12'' para Multifosfato Magnesiano.
\end{itemize}
\end{minipage}
\\\hline
\begin{minipage}[t]{0.47\columnwidth}
EXCEÇÃO
\end{minipage} & \begin{minipage}[t]{0.47\columnwidth}
\end{minipage}
\\\hline

\end{longtable}

 

\begin{longtable}[c]{@{}|p{4cm}|p{9cm}|@{}}
\hline
\begin{minipage}[t]{0.47\columnwidth}
\textbf{RS019}
\end{minipage} & \begin{minipage}[t]{0.47\columnwidth}
Excluir informações relacionadas à correção/recuperação do Fósforo
\end{minipage}
\\\hline
\begin{minipage}[t]{0.47\columnwidth}
SUMÁRIO
\end{minipage} & \begin{minipage}[t]{0.47\columnwidth}
O requisito é responsável pela remoção de valores relacionados à
correção/recuperação do Fósforo.
\end{minipage}
\\\hline
\begin{minipage}[t]{0.47\columnwidth}
PRÉ-CONDIÇÕES
\end{minipage} & \begin{minipage}[t]{0.47\columnwidth}
O usuário deve ter realizado o cadastro das informações relacionadas ao
Fósforo conforme descrito no requisito de sistema \textbf{RS018}.
\end{minipage}
\\\hline
\begin{minipage}[t]{0.47\columnwidth}
ATORES
\end{minipage} & \begin{minipage}[t]{0.47\columnwidth}
Usuário
\end{minipage}
\\\hline
\begin{minipage}[t]{0.47\columnwidth}
DESCRIÇÃO
\end{minipage} & \begin{minipage}[t]{0.47\columnwidth}
\begin{enumerate}
\def\labelenumi{\arabic{enumi}.}
\itemsep1pt\parskip0pt\parsep0pt
\item
  O usuário loga no sistema.
\item
  O sistema exibe uma tela com todos os registros já cadastrados
  relacionados ao Fósforo em forma de tabela.
\item
  O usuário clica no ícone com um ``X'' ao lado dos valores que deseja
  remover.
\item
  O sistema exibe uma mensagem de confirmação ao usuário com as opções
  de ``OK'' e ``CANCELAR''.
\item
  O usuário clica em OK. 
\item
  O sistema remove o cadastro referente.
\\\end{enumerate}
\end{minipage}
\\\hline
\begin{minipage}[t]{0.47\columnwidth}
ALTERNATIVAS
\end{minipage} & \begin{minipage}[t]{0.47\columnwidth}
\begin{enumerate}
\def\labelenumi{\arabic{enumi}.}
\itemsep1pt\parskip0pt\parsep0pt
\item
  Caso o usuário clique no botão ``CANCELAR'' descrito no passo 4, o
  sistema deve retornar ao passo 2.
\\\end{enumerate}
\end{minipage}
\\\hline
\begin{minipage}[t]{0.47\columnwidth}
EXCEÇÃO
\end{minipage} & \begin{minipage}[t]{0.47\columnwidth}
\begin{enumerate}
\def\labelenumi{\arabic{enumi}.}
\itemsep1pt\parskip0pt\parsep0pt
\item
  No ato da exclusão, uma mensagem de aviso deverá ser exibida,
  informando ao usuário que todos os dados que dependem das informações
  relacionadas à correção/recuperação do Fósforo serão excluídas.
\\\end{enumerate}
\end{minipage}
\\\hline

\end{longtable}

 

\begin{longtable}[c]{@{}|p{4cm}|p{9cm}|@{}}
\hline
\begin{minipage}[t]{0.47\columnwidth}
\textbf{RS020}
\end{minipage} & \begin{minipage}[t]{0.47\columnwidth}
Alterar informações relacionadas à correção/recuperação do
Fósforo
\end{minipage}
\\\hline
\begin{minipage}[t]{0.47\columnwidth}
SUMÁRIO
\end{minipage} & \begin{minipage}[t]{0.47\columnwidth}
O requisito é responsável pela alteração de valores relacionados à
correção/recuperação do Fósforo.
\end{minipage}
\\\hline
\begin{minipage}[t]{0.47\columnwidth}
PRÉ-CONDIÇÕES
\end{minipage} & \begin{minipage}[t]{0.47\columnwidth}
O usuário deve ter realizado o cadastro das informações relacionadas ao
Fósforo conforme descrito no requisito de sistema \textbf{RS018}.
\end{minipage}
\\\hline
\begin{minipage}[t]{0.47\columnwidth}
ATORES
\end{minipage} & \begin{minipage}[t]{0.47\columnwidth}
Usuário
\end{minipage}
\\\hline
\begin{minipage}[t]{0.47\columnwidth}
DESCRIÇÃO
\end{minipage} & \begin{minipage}[t]{0.47\columnwidth}
\begin{enumerate}
\def\labelenumi{\arabic{enumi}.}
\itemsep1pt\parskip0pt\parsep0pt
\item
  O usuário loga no sistema.
\item
  O sistema exibe uma tela com todos os registros já cadastrados
  relacionados ao Fósforo em forma de tabela.
\item
  O usuário clica em um ícone que representa a edição.
\item
  O mesmo formulário descrito no requisito de sistema \textbf{RS018}é exibido,
  com as informações do Fósforo podendo ser editadas.
\item
  O usuário faz as alterações. 
\item
  O usuário clica no botão confirmar.
\item
  O sistema registra as novas informações relacionadas ao Fósforo.
\\\end{enumerate}
\end{minipage}
\\\hline
\begin{minipage}[t]{0.47\columnwidth}
ALTERNATIVAS
\end{minipage} & \begin{minipage}[t]{0.47\columnwidth}
\end{minipage}
\\\hline
\begin{minipage}[t]{0.47\columnwidth}
EXCEÇÃO
\end{minipage} & \begin{minipage}[t]{0.47\columnwidth}
\begin{enumerate}
\def\labelenumi{\arabic{enumi}.}
\itemsep1pt\parskip0pt\parsep0pt
\item
  No ato da alteração, uma mensagem de aviso deverá ser exibida,
  informando ao usuário que todos os dados que dependem das informações
  relacionadas à correção/recuperação do Fósforo serão alteradas.
\\\end{enumerate}
\end{minipage}
\\\hline

\end{longtable}

 

\begin{longtable}[c]{@{}|p{4cm}|p{9cm}|@{}}
\hline
\begin{minipage}[t]{0.47\columnwidth}
\textbf{RS021}
\end{minipage} & \begin{minipage}[t]{0.47\columnwidth}
Listar informações relacionadas à correção/recuperação do Fósforo
\end{minipage}
\\\hline
\begin{minipage}[t]{0.47\columnwidth}
SUMÁRIO
\end{minipage} & \begin{minipage}[t]{0.47\columnwidth}
O requisito é responsável pela listagem de valores relacionados à
correção/recuperação do Fósforo.
\end{minipage}
\\\hline
\begin{minipage}[t]{0.47\columnwidth}
PRÉ-CONDIÇÕES
\end{minipage} & \begin{minipage}[t]{0.47\columnwidth}
O usuário deve ter realizado o cadastro das informações relacionadas ao
Fósforo conforme descrito no requisito de sistema \textbf{RS018}.
\end{minipage}
\\\hline
\begin{minipage}[t]{0.47\columnwidth}
ATORES
\end{minipage} & \begin{minipage}[t]{0.47\columnwidth}
Usuário
\end{minipage}
\\\hline
\begin{minipage}[t]{0.47\columnwidth}
DESCRIÇÃO
\end{minipage} & \begin{minipage}[t]{0.47\columnwidth}
\begin{enumerate}
\def\labelenumi{\arabic{enumi}.}
\itemsep1pt\parskip0pt\parsep0pt
\item
  O usuário loga no sistema.
\item
  O sistema exibe uma tela com todos os registros já cadastrados
  relacionados ao Fósforo em forma de tabela.
\item
  O usuário clica sobre o cadastro que deseja listas as informações.
\item
  O sistema exibe ao usuário as informações referentes ao cadastro
  selecionado.
\\\end{enumerate}
\end{minipage}
\\\hline
\begin{minipage}[t]{0.47\columnwidth}
ALTERNATIVAS
\end{minipage} & \begin{minipage}[t]{0.47\columnwidth}
\end{minipage}
\\\hline
\begin{minipage}[t]{0.47\columnwidth}
EXCEÇÃO
\end{minipage} & \begin{minipage}[t]{0.47\columnwidth}
\end{minipage}
\\\hline

\end{longtable}

 

\begin{longtable}[c]{@{}|p{4cm}|p{9cm}|@{}}
\hline
\begin{minipage}[t]{0.47\columnwidth}
\textbf{RS022}
\end{minipage} & \begin{minipage}[t]{0.47\columnwidth}
Exibir valores a serem aplicados no processo de correção/recuperação de
Fósforo
\end{minipage}
\\\hline
\begin{minipage}[t]{0.47\columnwidth}
SUMÁRIO
\end{minipage} & \begin{minipage}[t]{0.47\columnwidth}
O requisito é responsável por exibir valores úteis para o processo de
correção/recuperação do Fósforo.
\end{minipage}
\\\hline
\begin{minipage}[t]{0.47\columnwidth}
PRÉ-CONDIÇÕES
\end{minipage} & \begin{minipage}[t]{0.47\columnwidth}
O usuário deve ter inserido as informações quanto ao teor de Fósforo a
ser atingido, fonte de Fósforo a utilizar e eficiência do Fósforo
conforme descrito no requisito de sistema \textbf{RS018}.
\end{minipage}
\\\hline
\begin{minipage}[t]{0.47\columnwidth}
ATORES
\end{minipage} & \begin{minipage}[t]{0.47\columnwidth}
Sistema
\end{minipage}
\\\hline
\begin{minipage}[t]{0.47\columnwidth}
DESCRIÇÃO
\end{minipage} & \begin{minipage}[t]{0.47\columnwidth}
\begin{enumerate}
\def\labelenumi{\arabic{enumi}.}
\itemsep1pt\parskip0pt\parsep0pt
\item
  O usuário loga no sistema.
\item
  O usuário preenche os dados relacionados ao processo de
  correção/recuperação do Fósforo.
\item
  O sistema exibe logo depois do formulário a quantidade de Fósforo a
  ser aplicada (em kg/hectare) e também seu custo (em R\$/ha).
\\\end{enumerate}
\end{minipage}
\\\hline
\begin{minipage}[t]{0.47\columnwidth}
ALTERNATIVAS
\end{minipage} & \begin{minipage}[t]{0.47\columnwidth}
\end{minipage}
\\\hline
\begin{minipage}[t]{0.47\columnwidth}
EXCEÇÃO
\end{minipage} & \begin{minipage}[t]{0.47\columnwidth}
Nenhum valor será exibido caso o usuário tenha deixado de preencher
algum campo conforme descrito no requisito de sistema \textbf{RS018}.
\end{minipage}
\\\hline

\end{longtable}

 

\begin{longtable}[c]{@{}|p{4cm}|p{9cm}|@{}}
\hline
\begin{minipage}[t]{0.47\columnwidth}
\textbf{RS023}
\end{minipage} & \begin{minipage}[t]{0.47\columnwidth}
Cadastrarvalor/ton. (R\$) no processo relacionado às fontes de Fósforo
\end{minipage}
\\\hline
\begin{minipage}[t]{0.47\columnwidth}
SUMÁRIO
\end{minipage} & \begin{minipage}[t]{0.47\columnwidth}
O requisito é responsável por inserir dados referente às fontes de
Fósforo.
\end{minipage}
\\\hline
\begin{minipage}[t]{0.47\columnwidth}
PRÉ-CONDIÇÕES
\end{minipage} & \begin{minipage}[t]{0.47\columnwidth}
\end{minipage}
\\\hline
\begin{minipage}[t]{0.47\columnwidth}
ATORES
\end{minipage} & \begin{minipage}[t]{0.47\columnwidth}
Usuário
\end{minipage}
\\\hline
\begin{minipage}[t]{0.47\columnwidth}
DESCRIÇÃO
\end{minipage} & \begin{minipage}[t]{0.47\columnwidth}
\begin{enumerate}
\def\labelenumi{\arabic{enumi}.}
\itemsep1pt\parskip0pt\parsep0pt
\item
  O usuário loga no sistema.
\item
  O sistema exibe uma tela com botões referentes ao gerenciamento de
  informações relacionadas às fontes do Fósforo.
\item
  O usuário clica no botão ``Cadastrar fontes de Fósforo''.
\item
  O usuário insere valores relacionados às fontes de Fósforo tais como:
\\\end{enumerate}

\begin{itemize}
\itemsep1pt\parskip0pt\parsep0pt
\item
  Superfosfato simples;
\item
  Superfosfato triplo;
\item
  MAP;
\item
  DAP;
\item
  Yoorin;
\item
  Fosfato Arad;
\item
  Fosfato Gafsa;
\item
  Fosfato Daoui;
\item
  Fosfato Patos Minas;
\item
  Escória de Thomas;
\item
  Ácido Fosfórico;
\item
  Multifosfato magnesiano.
\end{itemize}

\begin{enumerate}
\def\labelenumi{\arabic{enumi}.}
\setcounter{enumi}{4}
\itemsep1pt\parskip0pt\parsep0pt
\item
  O usuário clica no botão continuar.
\item
  O sistema registra a inserção dos valores.
\\\end{enumerate}
\end{minipage}
\\\hline
\begin{minipage}[t]{0.47\columnwidth}
ALTERNATIVAS
\end{minipage} & \begin{minipage}[t]{0.47\columnwidth}
\end{minipage}
\\\hline
\begin{minipage}[t]{0.47\columnwidth}
EXCEÇÃO
\end{minipage} & \begin{minipage}[t]{0.47\columnwidth}
\end{minipage}
\\\hline

\end{longtable}

 

\begin{longtable}[c]{@{}|p{4cm}|p{9cm}|@{}}
\hline
\begin{minipage}[t]{0.47\columnwidth}
\textbf{RS024}
\end{minipage} & \begin{minipage}[t]{0.47\columnwidth}
Excluirvalor/ton. (R\$) no processo relacionado às fontes de Fósforo
\end{minipage}
\\\hline
\begin{minipage}[t]{0.47\columnwidth}
SUMÁRIO
\end{minipage} & \begin{minipage}[t]{0.47\columnwidth}
O requisito é responsável por remover dados referentes às fontes de
Fósforo.
\end{minipage}
\\\hline
\begin{minipage}[t]{0.47\columnwidth}
PRÉ-CONDIÇÕES
\end{minipage} & \begin{minipage}[t]{0.47\columnwidth}
O usuário deve ter realizado o cadastro das informações
relacionadas às fontes de Fósforo conforme descrito no requisito
de sistema \textbf{RS023}.
\end{minipage}
\\\hline
\begin{minipage}[t]{0.47\columnwidth}
ATORES
\end{minipage} & \begin{minipage}[t]{0.47\columnwidth}
Usuário
\end{minipage}
\\\hline
\begin{minipage}[t]{0.47\columnwidth}
DESCRIÇÃO
\end{minipage} & \begin{minipage}[t]{0.47\columnwidth}
\begin{enumerate}
\def\labelenumi{\arabic{enumi}.}
\itemsep1pt\parskip0pt\parsep0pt
\item
  O usuário loga no sistema.
\item
  O sistema exibe uma tela com todos os registros já cadastrados
  relacionados às fontes de Fósforo em forma de tabela.
\item
  O usuário clica no ícone com um ``X'' ao lado dos valores que deseja
  remover. 
\item
  O sistema exibe uma mensagem de confirmação ao usuário com as opções
  ``OK'' e ``CANCELAR''.
\item
  O usuário clica em OK.
\item
  O sistema remove o cadastro referente.
\\\end{enumerate}
\end{minipage}
\\\hline
\begin{minipage}[t]{0.47\columnwidth}
ALTERNATIVAS
\end{minipage} & \begin{minipage}[t]{0.47\columnwidth}
\begin{enumerate}
\def\labelenumi{\arabic{enumi}.}
\itemsep1pt\parskip0pt\parsep0pt
\item
  Caso o usuário clique no botão ``CANCELAR'' descrito no passo 4, o
  sistema executa o passo 2.
\\\end{enumerate}
\end{minipage}
\\\hline
\begin{minipage}[t]{0.47\columnwidth}
EXCEÇÃO
\end{minipage} & \begin{minipage}[t]{0.47\columnwidth}
\begin{enumerate}
\def\labelenumi{\arabic{enumi}.}
\itemsep1pt\parskip0pt\parsep0pt
\item
  No ato da exclusão, uma mensagem de aviso deverá ser exibida,
  informando ao usuário que todos os dados que dependem das informações
  relacionadas às fontes de Fósforo serão excluídas.
\\\end{enumerate}
\end{minipage}
\\\hline

\end{longtable}

 

\begin{longtable}[c]{@{}|p{4cm}|p{9cm}|@{}}
\hline
\begin{minipage}[t]{0.47\columnwidth}
\textbf{RS025}
\end{minipage} & \begin{minipage}[t]{0.47\columnwidth}
Alterarvalor/ton. (R\$) no processo relacionado às fontes de Fósforo
\end{minipage}
\\\hline
\begin{minipage}[t]{0.47\columnwidth}
SUMÁRIO
\end{minipage} & \begin{minipage}[t]{0.47\columnwidth}
O requisito é responsável por alterar dados referentes às fontes de
Fósforo.
\end{minipage}
\\\hline
\begin{minipage}[t]{0.47\columnwidth}
PRÉ-CONDIÇÕES
\end{minipage} & \begin{minipage}[t]{0.47\columnwidth}
O usuário deve ter realizado o cadastro das informações relacionadas às
fontes de Fósforo conforme descrito no requisito de sistema \textbf{RS023}.
\end{minipage}
\\\hline
\begin{minipage}[t]{0.47\columnwidth}
ATORES
\end{minipage} & \begin{minipage}[t]{0.47\columnwidth}
Usuário
\end{minipage}
\\\hline
\begin{minipage}[t]{0.47\columnwidth}
DESCRIÇÃO
\end{minipage} & \begin{minipage}[t]{0.47\columnwidth}
\begin{enumerate}
\def\labelenumi{\arabic{enumi}.}
\itemsep1pt\parskip0pt\parsep0pt
\item
  O usuário loga no sistema.
\item
  O sistema exibe uma tela com todos os registros já cadastrados
  relacionados às fontes de Fósforo em forma de tabela.
\item
  O usuário clica em um ícone que representa a edição.
\item
  O mesmo formulário descrito no requisito de sistema \textbf{RS023}é exibido,
  com as informações das fontes de Fósforo podendo ser editadas.
\item
  O usuário faz as alterações.
\item
  O usuário clica no botão confirmar.
\item
  O sistema registra as novas informações relacionadas às fontes de
  Fósforo.
\\\end{enumerate}
\end{minipage}
\\\hline
\begin{minipage}[t]{0.47\columnwidth}
ALTERNATIVAS
\end{minipage} & \begin{minipage}[t]{0.47\columnwidth}
\end{minipage}
\\\hline
\begin{minipage}[t]{0.47\columnwidth}
EXCEÇÃO
\end{minipage} & \begin{minipage}[t]{0.47\columnwidth}
\begin{enumerate}
\def\labelenumi{\arabic{enumi}.}
\itemsep1pt\parskip0pt\parsep0pt
\item
  No ato da alteração, uma mensagem de aviso deverá ser exibida,
  informando ao usuário que todos os dados que dependem das informações
  relacionadas às fontes de Fósforo serão alteradas.
\\\end{enumerate}
\end{minipage}
\\\hline

\end{longtable}

 

\begin{longtable}[c]{@{}|p{4cm}|p{9cm}|@{}}
\hline
\begin{minipage}[t]{0.47\columnwidth}
\textbf{RS026}
\end{minipage} & \begin{minipage}[t]{0.47\columnwidth}
Listar informações relacionadas às fontes de Fósforo
\end{minipage}
\\\hline
\begin{minipage}[t]{0.47\columnwidth}
SUMÁRIO
\end{minipage} & \begin{minipage}[t]{0.47\columnwidth}
O requisito é responsável pela listagem de valores relacionados às
fontes de Fósforo.
\end{minipage}
\\\hline
\begin{minipage}[t]{0.47\columnwidth}
PRÉ-CONDIÇÕES
\end{minipage} & \begin{minipage}[t]{0.47\columnwidth}
O usuário deve ter realizado o cadastro das informações relacionadas às
fontes de Fósforo conforme descrito no requisito de sistema \textbf{RS023}.
\end{minipage}
\\\hline
\begin{minipage}[t]{0.47\columnwidth}
ATORES
\end{minipage} & \begin{minipage}[t]{0.47\columnwidth}
Usuário
\end{minipage}
\\\hline
\begin{minipage}[t]{0.47\columnwidth}
DESCRIÇÃO
\end{minipage} & \begin{minipage}[t]{0.47\columnwidth}
\begin{enumerate}
\def\labelenumi{\arabic{enumi}.}
\itemsep1pt\parskip0pt\parsep0pt
\item
  O usuário loga no sistema.
\item
  O sistema exibe uma tela com todos os registros já cadastrados
  relacionados às fontes de Fósforo em forma de tabela.
\item
  O usuário clica sobre o cadastro que deseja listar as informações.
\item
  O sistema exibe ao usuário as informações referentes ao cadastro
  selecionado.
\\\end{enumerate}
\end{minipage}
\\\hline
\begin{minipage}[t]{0.47\columnwidth}
ALTERNATIVAS
\end{minipage} & \begin{minipage}[t]{0.47\columnwidth}
\end{minipage}
\\\hline
\begin{minipage}[t]{0.47\columnwidth}
EXCEÇÃO
\end{minipage} & \begin{minipage}[t]{0.47\columnwidth}
\end{minipage}
\\\hline

\end{longtable}

 

\begin{longtable}[c]{@{}|p{4cm}|p{9cm}|@{}}
\hline
\begin{minipage}[t]{0.47\columnwidth}
\textbf{RS027}
\end{minipage} & \begin{minipage}[t]{0.47\columnwidth}
Cadastrar informações relacionadas à correção/recuperação de Potássio
\end{minipage}
\\\hline
\begin{minipage}[t]{0.47\columnwidth}
SUMÁRIO
\end{minipage} & \begin{minipage}[t]{0.47\columnwidth}
O requisito é responsável pela inserção de valores relacionados à
correção/recuperação de Potássio.
\end{minipage}
\\\hline
\begin{minipage}[t]{0.47\columnwidth}
PRÉ-CONDIÇÕES
\end{minipage} & \begin{minipage}[t]{0.47\columnwidth}
\end{minipage}
\\\hline
\begin{minipage}[t]{0.47\columnwidth}
ATORES
\end{minipage} & \begin{minipage}[t]{0.47\columnwidth}
Usuário
\end{minipage}
\\\hline
\begin{minipage}[t]{0.47\columnwidth}
DESCRIÇÃO
\end{minipage} & \begin{minipage}[t]{0.47\columnwidth}
\begin{enumerate}
\def\labelenumi{\arabic{enumi}.}
\itemsep1pt\parskip0pt\parsep0pt
\item
  O usuário loga no sistema.
\item
  O sistema exibe uma tela com botões referentes ao gerenciamento de
  informações relacionadas à correção/recuperação de Potássio.
\item
  O usuário clica no botão ``Cadastrar correção/recuperação de
  Potássio''.
\item
  O usuário insere valores relacionados ao Potássio tais como:
\\\end{enumerate}

\begin{itemize}
\itemsep1pt\parskip0pt\parsep0pt
\item
  participação do Potássio na CTC;
\item
  fonte de Potássio a ser utilizada.
\end{itemize}

\begin{enumerate}
\def\labelenumi{\arabic{enumi}.}
\setcounter{enumi}{4}
\itemsep1pt\parskip0pt\parsep0pt
\item
  O usuário clica em continuar.
\item
  O sistema registra a inserção dos valores.
\\\end{enumerate}
\end{minipage}
\\\hline
\begin{minipage}[t]{0.47\columnwidth}
ALTERNATIVAS
\end{minipage} & \begin{minipage}[t]{0.47\columnwidth}
\begin{enumerate}
\def\labelenumi{\arabic{enumi}.}
\itemsep1pt\parskip0pt\parsep0pt
\item
  No campo referente à fonte de Potássio a ser utilizada, o valor
  informado deve ser um número inteiro dentro do intervalo de 1 à 3,
  onde:
\\\end{enumerate}

\begin{itemize}
\itemsep1pt\parskip0pt\parsep0pt
\item
  ``1'' representa Cloreto de Potássio;
\item
  ``2'' representa Sulfato de Potássio.
\item
  ``3'' representa Sulfato de Potássio e Magnésio.
\end{itemize}
\end{minipage}
\\\hline
\begin{minipage}[t]{0.47\columnwidth}
EXCEÇÃO
\end{minipage} & \begin{minipage}[t]{0.47\columnwidth}
\end{minipage}
\\\hline

\end{longtable}

 

\begin{longtable}[c]{@{}|p{4cm}|p{9cm}|@{}}
\hline
\begin{minipage}[t]{0.47\columnwidth}
\textbf{RS028}
\end{minipage} & \begin{minipage}[t]{0.47\columnwidth}
Excluir informações relacionadas à correção/recuperação de Potássio
\end{minipage}
\\\hline
\begin{minipage}[t]{0.47\columnwidth}
SUMÁRIO
\end{minipage} & \begin{minipage}[t]{0.47\columnwidth}
O requisito é responsável pela remoção de valores relacionados à
correção/recuperação de Potássio.
\end{minipage}
\\\hline
\begin{minipage}[t]{0.47\columnwidth}
PRÉ-CONDIÇÕES
\end{minipage} & \begin{minipage}[t]{0.47\columnwidth}
O usuário deve ter realizado o cadastro das informações relacionadas ao
Fósforo conforme descrito no requisito de sistema \textbf{RS027}.
\end{minipage}
\\\hline
\begin{minipage}[t]{0.47\columnwidth}
ATORES
\end{minipage} & \begin{minipage}[t]{0.47\columnwidth}
Usuário
\end{minipage}
\\\hline
\begin{minipage}[t]{0.47\columnwidth}
DESCRIÇÃO
\end{minipage} & \begin{minipage}[t]{0.47\columnwidth}
\begin{enumerate}
\def\labelenumi{\arabic{enumi}.}
\itemsep1pt\parskip0pt\parsep0pt
\item
  O usuário loga no sistema.
\item
  O sistema exibe uma tela com todos os registros já cadastrados
  relacionados ao Potássio em forma de tabela.
\item
  O usuário clica no ícone com um ``X'' ao lado dos valores que deseja
  remover.
\item
  O sistema exibe uma mensagem de confirmação ao usuário com as opções
  de ``OK'' e ``CANCELAR''.
\item
  O usuário clica em OK.
\item
  O sistema remove o cadastro referente.
\\\end{enumerate}
\end{minipage}
\\\hline
\begin{minipage}[t]{0.47\columnwidth}
ALTERNATIVAS
\end{minipage} & \begin{minipage}[t]{0.47\columnwidth}
\begin{enumerate}
\def\labelenumi{\arabic{enumi}.}
\itemsep1pt\parskip0pt\parsep0pt
\item
  No passo 4, caso o usuário clique em ``CANCELAR'', o sistema executa o
  passo 2.
\\\end{enumerate}
\end{minipage}
\\\hline
\begin{minipage}[t]{0.47\columnwidth}
EXCEÇÃO
\end{minipage} & \begin{minipage}[t]{0.47\columnwidth}
\begin{enumerate}
\def\labelenumi{\arabic{enumi}.}
\itemsep1pt\parskip0pt\parsep0pt
\item
  No ato da exclusão, uma mensagem de aviso deverá ser exibida,
  informando ao usuário que todos os dados que dependem das informações
  relacionadas à correção/recuperação do Potássio serão excluídas.
\\\end{enumerate}
\end{minipage}
\\\hline

\end{longtable}

 

\begin{longtable}[c]{@{}|p{4cm}|p{9cm}|@{}}
\hline
\begin{minipage}[t]{0.47\columnwidth}
\textbf{RS029}
\end{minipage} & \begin{minipage}[t]{0.47\columnwidth}
Alterar informações relacionadas à correção/recuperação de
Potássio
\end{minipage}
\\\hline
\begin{minipage}[t]{0.47\columnwidth}
SUMÁRIO
\end{minipage} & \begin{minipage}[t]{0.47\columnwidth}
O requisito é responsável pela alteração de valores relacionados à
correção/recuperação de Potássio.
\end{minipage}
\\\hline
\begin{minipage}[t]{0.47\columnwidth}
PRÉ-CONDIÇÕES
\end{minipage} & \begin{minipage}[t]{0.47\columnwidth}
O usuário deve ter realizado o cadastro das informações relacionadas ao
Potássio conforme descrito no requisito de sistema \textbf{RS027}.
\end{minipage}
\\\hline
\begin{minipage}[t]{0.47\columnwidth}
ATORES
\end{minipage} & \begin{minipage}[t]{0.47\columnwidth}
Usuário
\end{minipage}
\\\hline
\begin{minipage}[t]{0.47\columnwidth}
DESCRIÇÃO
\end{minipage} & \begin{minipage}[t]{0.47\columnwidth}
\begin{enumerate}
\def\labelenumi{\arabic{enumi}.}
\itemsep1pt\parskip0pt\parsep0pt
\item
  O usuário loga no sistema.
\item
  O sistema exibe uma tela com todos os registros já cadastrados
  relacionados ao Potássio em forma de tabela.
\item
  O usuário clica em um ícone que representa a edição.
\item
  O mesmo formulário descrito no requisito de sistema \textbf{RS027}é exibido,
  com as informações do Potássio podendo ser editadas.
\item
  O usuário faz as alterações.
\item
  O usuário clica no botão confirmar.
\item
  O sistema registra as novas informações relacionadas ao Potássio.
\\\end{enumerate}
\end{minipage}
\\\hline
\begin{minipage}[t]{0.47\columnwidth}
ALTERNATIVAS
\end{minipage} & \begin{minipage}[t]{0.47\columnwidth}
\end{minipage}
\\\hline
\begin{minipage}[t]{0.47\columnwidth}
EXCEÇÃO
\end{minipage} & \begin{minipage}[t]{0.47\columnwidth}
\begin{enumerate}
\def\labelenumi{\arabic{enumi}.}
\itemsep1pt\parskip0pt\parsep0pt
\item
  No ato da alteração, uma mensagem de aviso deverá ser exibida,
  informando ao usuário que todos os dados que dependem das informações
  relacionadas à correção/recuperação do Potássio serão alteradas.
\\\end{enumerate}
\end{minipage}
\\\hline

\end{longtable}

 

\begin{longtable}[c]{@{}|p{4cm}|p{9cm}|@{}}
\hline
\begin{minipage}[t]{0.47\columnwidth}
\textbf{RS030}
\end{minipage} & \begin{minipage}[t]{0.47\columnwidth}
Listar informações relacionadas à correção/recuperação de Potássio
\end{minipage}
\\\hline
\begin{minipage}[t]{0.47\columnwidth}
SUMÁRIO
\end{minipage} & \begin{minipage}[t]{0.47\columnwidth}
O requisito é responsável pela listagem de valores relacionados à
correção/recuperação de Potássio.
\end{minipage}
\\\hline
\begin{minipage}[t]{0.47\columnwidth}
PRÉ-CONDIÇÕES
\end{minipage} & \begin{minipage}[t]{0.47\columnwidth}
O usuário deve ter realizado o cadastro das informações relacionadas ao
Potássio conforme descrito no requisito de sistema \textbf{RS027}.
\end{minipage}
\\\hline
\begin{minipage}[t]{0.47\columnwidth}
ATORES
\end{minipage} & \begin{minipage}[t]{0.47\columnwidth}
Usuário
\end{minipage}
\\\hline
\begin{minipage}[t]{0.47\columnwidth}
DESCRIÇÃO
\end{minipage} & \begin{minipage}[t]{0.47\columnwidth}
\begin{enumerate}
\def\labelenumi{\arabic{enumi}.}
\itemsep1pt\parskip0pt\parsep0pt
\item
  O usuário loga no sistema.
\item
  O sistema exibe uma tela com todos os registros já cadastrados
  relacionados ao Potássio em forma de tabela.
\item
  O usuário clica sobre o cadastro que deseja listas as informações.
\item
  O sistema exibe ao usuário as informações referentes ao cadastro
  selecionado.
\\\end{enumerate}
\end{minipage}
\\\hline
\begin{minipage}[t]{0.47\columnwidth}
ALTERNATIVAS
\end{minipage} & \begin{minipage}[t]{0.47\columnwidth}
\end{minipage}
\\\hline
\begin{minipage}[t]{0.47\columnwidth}
EXCEÇÃO
\end{minipage} & \begin{minipage}[t]{0.47\columnwidth}
\end{minipage}
\\\hline

\end{longtable}

 

\begin{longtable}[c]{@{}|p{4cm}|p{9cm}|@{}}
\hline
\begin{minipage}[t]{0.47\columnwidth}
\textbf{RS031}
\end{minipage} & \begin{minipage}[t]{0.47\columnwidth}
Exibir valores relacionados à correção/recuperação do Potássio
\end{minipage}
\\\hline
\begin{minipage}[t]{0.47\columnwidth}
SUMÁRIO
\end{minipage} & \begin{minipage}[t]{0.47\columnwidth}
O requisito é responsável por dispor dados que dizem respeito à
correção/recuperação do Potássio.
\end{minipage}
\\\hline
\begin{minipage}[t]{0.47\columnwidth}
PRÉ-CONDIÇÕES
\end{minipage} & \begin{minipage}[t]{0.47\columnwidth}
O usuário deve ter inserido as informações do Potássio conforme descrito
no requisito de sistema \textbf{RS027}.
\end{minipage}
\\\hline
\begin{minipage}[t]{0.47\columnwidth}
ATORES
\end{minipage} & \begin{minipage}[t]{0.47\columnwidth}
Usuário
\end{minipage}
\\\hline
\begin{minipage}[t]{0.47\columnwidth}
DESCRIÇÃO
\end{minipage} & \begin{minipage}[t]{0.47\columnwidth}
\begin{enumerate}
\def\labelenumi{\arabic{enumi}.}
\itemsep1pt\parskip0pt\parsep0pt
\item
  O usuário loga no sistema.
\item
  O usuário preenche os dados relacionados ao processo de
  correção/recuperação do Potássio.
\item
  O sistema exibe dados relacionados ao percentual de participação atual
  do Potássio na CTC, percentual de participação do Potássio na CTC após
  correção, percentual de participação ideal do Potássio na CTC e a
  fonte de Potássio a ser utilizada.
\\\end{enumerate}
\end{minipage}
\\\hline
\begin{minipage}[t]{0.47\columnwidth}
ALTERNATIVAS
\end{minipage} & \begin{minipage}[t]{0.47\columnwidth}
\end{minipage}
\\\hline
\begin{minipage}[t]{0.47\columnwidth}
EXCEÇÃO
\end{minipage} & \begin{minipage}[t]{0.47\columnwidth}
\end{minipage}
\\\hline

\end{longtable}

 

\begin{longtable}[c]{@{}|p{4cm}|p{9cm}|@{}}
\hline
\begin{minipage}[t]{0.47\columnwidth}
\textbf{RS032}
\end{minipage} & \begin{minipage}[t]{0.47\columnwidth}
Cadastrarvalor/ton. (R\$) no processo relacionado às fontes de Potássio
\end{minipage}
\\\hline
\begin{minipage}[t]{0.47\columnwidth}
SUMÁRIO
\end{minipage} & \begin{minipage}[t]{0.47\columnwidth}
O requisito é responsável por inserir dados referente às fontes de
Potássio.
\end{minipage}
\\\hline
\begin{minipage}[t]{0.47\columnwidth}
PRÉ-CONDIÇÕES
\end{minipage} & \begin{minipage}[t]{0.47\columnwidth}
\end{minipage}
\\\hline
\begin{minipage}[t]{0.47\columnwidth}
ATORES
\end{minipage} & \begin{minipage}[t]{0.47\columnwidth}
Usuário
\end{minipage}
\\\hline
\begin{minipage}[t]{0.47\columnwidth}
DESCRIÇÃO
\end{minipage} & \begin{minipage}[t]{0.47\columnwidth}
\begin{enumerate}
\def\labelenumi{\arabic{enumi}.}
\itemsep1pt\parskip0pt\parsep0pt
\item
  O usuário loga no sistema.
\item
  O sistema exibe uma tela com botões referentes ao gerenciamento de
  informações relacionadas às fontes de Potássio.
\item
  O usuário clica no botão ``Cadastrar fontes de Potássio''.
\item
  O usuário insere valores relacionados às fontes de Potássio tais como:
\\\end{enumerate}

\begin{itemize}
\itemsep1pt\parskip0pt\parsep0pt
\item
  Cloreto de Potássio;
\item
  Sulfato de Potássio;
\item
  Sulfato de Potássio/Magnésio.
\end{itemize}

\begin{enumerate}
\def\labelenumi{\arabic{enumi}.}
\setcounter{enumi}{4}
\itemsep1pt\parskip0pt\parsep0pt
\item
  O usuário clica no botão continuar.
\item
  O sistema registra a inserção dos valores.
\\\end{enumerate}
\end{minipage}
\\\hline
\begin{minipage}[t]{0.47\columnwidth}
ALTERNATIVAS
\end{minipage} & \begin{minipage}[t]{0.47\columnwidth}
\end{minipage}
\\\hline
\begin{minipage}[t]{0.47\columnwidth}
EXCEÇÃO
\end{minipage} & \begin{minipage}[t]{0.47\columnwidth}
\end{minipage}
\\\hline

\end{longtable}

 

\begin{longtable}[c]{@{}|p{4cm}|p{9cm}|@{}}
\hline
\begin{minipage}[t]{0.47\columnwidth}
\textbf{RS033}
\end{minipage} & \begin{minipage}[t]{0.47\columnwidth}
Excluirvalor/ton. (R\$) no processo relacionado às fontes de Potássio
\end{minipage}
\\\hline
\begin{minipage}[t]{0.47\columnwidth}
SUMÁRIO
\end{minipage} & \begin{minipage}[t]{0.47\columnwidth}
O requisito é responsável por remover dados referentes às fontes de
Potássio.
\end{minipage}
\\\hline
\begin{minipage}[t]{0.47\columnwidth}
PRÉ-CONDIÇÕES
\end{minipage} & \begin{minipage}[t]{0.47\columnwidth}
O usuário deve ter realizado o cadastro das informações relacionadas às
fontes de Potássio conforme descrito no requisito de sistema \textbf{RS032}.
\end{minipage}
\\\hline
\begin{minipage}[t]{0.47\columnwidth}
ATORES
\end{minipage} & \begin{minipage}[t]{0.47\columnwidth}
Usuário
\end{minipage}
\\\hline
\begin{minipage}[t]{0.47\columnwidth}
DESCRIÇÃO
\end{minipage} & \begin{minipage}[t]{0.47\columnwidth}
\begin{enumerate}
\def\labelenumi{\arabic{enumi}.}
\itemsep1pt\parskip0pt\parsep0pt
\item
  O usuário loga no sistema.
\item
  O sistema exibe uma tela com todos os registros já cadastrados
  relacionados às fontes de Potássio em forma de tabela.
\item
  O usuário clica no ícone com um ``X'' ao lado dos valores que deseja
  remover.
\item
  O sistema exibe uma mensagem de confirmação ao usuário com as opções
  ``OK'' e ``CANCELAR''.
\item
  O usuário clica em OK.
\item
  O sistema remove o cadastro referente.
\\\end{enumerate}
\end{minipage}
\\\hline
\begin{minipage}[t]{0.47\columnwidth}
ALTERNATIVAS
\end{minipage} & \begin{minipage}[t]{0.47\columnwidth}
\begin{enumerate}
\def\labelenumi{\arabic{enumi}.}
\itemsep1pt\parskip0pt\parsep0pt
\item
  Caso o usuário clique no botão ``CANCELAR'' descrito no passo 4, o
  sistema executa o passo 2.
\\\end{enumerate}
\end{minipage}
\\\hline
\begin{minipage}[t]{0.47\columnwidth}
EXCEÇÃO
\end{minipage} & \begin{minipage}[t]{0.47\columnwidth}
\begin{enumerate}
\def\labelenumi{\arabic{enumi}.}
\itemsep1pt\parskip0pt\parsep0pt
\item
  No ato da exclusão, uma mensagem de aviso deverá ser exibida,
  informando ao usuário que todos os dados que dependem das informações
  relacionadas às fontes de Potássio serão excluídas.
\\\end{enumerate}
\end{minipage}
\\\hline

\end{longtable}

 

\begin{longtable}[c]{@{}|p{4cm}|p{9cm}|@{}}
\hline
\begin{minipage}[t]{0.47\columnwidth}
\textbf{RS034}
\end{minipage} & \begin{minipage}[t]{0.47\columnwidth}
Alterarvalor/ton. (R\$) no processo relacionado às fontes de Potássio
\end{minipage}
\\\hline
\begin{minipage}[t]{0.47\columnwidth}
SUMÁRIO
\end{minipage} & \begin{minipage}[t]{0.47\columnwidth}
O requisito é responsável por alterar dados referentes às fontes de
Potássio.
\end{minipage}
\\\hline
\begin{minipage}[t]{0.47\columnwidth}
PRÉ-CONDIÇÕES
\end{minipage} & \begin{minipage}[t]{0.47\columnwidth}
O usuário deve ter realizado o cadastro das informações relacionadas às
fontes de Potássio conforme descrito no requisito de sistema \textbf{RS032}.
\end{minipage}
\\\hline
\begin{minipage}[t]{0.47\columnwidth}
ATORES
\end{minipage} & \begin{minipage}[t]{0.47\columnwidth}
Usuário
\end{minipage}
\\\hline
\begin{minipage}[t]{0.47\columnwidth}
DESCRIÇÃO
\end{minipage} & \begin{minipage}[t]{0.47\columnwidth}
\begin{enumerate}
\def\labelenumi{\arabic{enumi}.}
\itemsep1pt\parskip0pt\parsep0pt
\item
  O usuário loga no sistema.
\item
  O sistema exibe uma tela com todos os registros já cadastrados
  relacionados às fontes de Potássio em forma de tabela.
\item
  O usuário clica em um ícone que representa a edição.
\item
  O mesmo formulário descrito no requisito de sistema \textbf{RS032} é exibido,
  com as informações das fontes de Potássio podendo ser editadas.
\item
  O usuário faz as alterações.
\item
  O usuário clica no botão confirmar.
\item
  O sistema registra as novas informações relacionadas às fontes de
  Potássio.
\\\end{enumerate}
\end{minipage}
\\\hline
\begin{minipage}[t]{0.47\columnwidth}
ALTERNATIVAS
\end{minipage} & \begin{minipage}[t]{0.47\columnwidth}
\end{minipage}
\\\hline
\begin{minipage}[t]{0.47\columnwidth}
EXCEÇÃO
\end{minipage} & \begin{minipage}[t]{0.47\columnwidth}
\begin{enumerate}
\def\labelenumi{\arabic{enumi}.}
\itemsep1pt\parskip0pt\parsep0pt
\item
  No ato da alteração, uma mensagem de aviso deverá ser exibida,
  informando ao usuário que todos os dados que dependem das informações
  relacionadas às fontes de Potássio serão alteração.
\\\end{enumerate}
\end{minipage}
\\\hline

\end{longtable}

 

\begin{longtable}[c]{@{}|p{4cm}|p{9cm}|@{}}
\hline
\begin{minipage}[t]{0.47\columnwidth}
\textbf{RS035}
\end{minipage} & \begin{minipage}[t]{0.47\columnwidth}
Listar informações relacionadas às fontes de Potássio
\end{minipage}
\\\hline
\begin{minipage}[t]{0.47\columnwidth}
SUMÁRIO
\end{minipage} & \begin{minipage}[t]{0.47\columnwidth}
O requisito é responsável pela listagem de valores relacionados às
fontes de Potássio.
\end{minipage}
\\\hline
\begin{minipage}[t]{0.47\columnwidth}
PRÉ-CONDIÇÕES
\end{minipage} & \begin{minipage}[t]{0.47\columnwidth}
O usuário deve ter realizado o cadastro das informações relacionadas às
fontes de Potássio conforme descrito no requisito de sistema \textbf{RS032}.
\end{minipage}
\\\hline
\begin{minipage}[t]{0.47\columnwidth}
ATORES
\end{minipage} & \begin{minipage}[t]{0.47\columnwidth}
Usuário
\end{minipage}
\\\hline
\begin{minipage}[t]{0.47\columnwidth}
DESCRIÇÃO
\end{minipage} & \begin{minipage}[t]{0.47\columnwidth}
\begin{enumerate}
\def\labelenumi{\arabic{enumi}.}
\itemsep1pt\parskip0pt\parsep0pt
\item
  O usuário loga no sistema.
\item
  O sistema exibe uma tela com todos os registros já cadastrados
  relacionados às fontes de Potássio em forma de tabela.
\item
  O usuário clica sobre o cadastro que deseja listar as informações.
\item
  O sistema exibe ao usuário as informações referentes ao cadastro
  selecionado.
\\\end{enumerate}
\end{minipage}
\\\hline
\begin{minipage}[t]{0.47\columnwidth}
ALTERNATIVAS
\end{minipage} & \begin{minipage}[t]{0.47\columnwidth}
\end{minipage}
\\\hline
\begin{minipage}[t]{0.47\columnwidth}
EXCEÇÃO
\end{minipage} & \begin{minipage}[t]{0.47\columnwidth}
\end{minipage}
\\\hline

\end{longtable}



\begin{longtable}[c]{@{}|p{4cm}|p{9cm}|@{}}
\hline
\begin{minipage}[t]{0.47\columnwidth}
\textbf{RS036}
\end{minipage} & \begin{minipage}[t]{0.47\columnwidth}
Informar dados para a correção do cálcio e magnésio no solo
\end{minipage}
\\\hline
\begin{minipage}[t]{0.47\columnwidth}
SUMÁRIO
\end{minipage} & \begin{minipage}[t]{0.47\columnwidth}
O sistema deve permitir o usuário informar dados acerca da correção do
cálcio no solo.
\end{minipage}
\\\hline
\begin{minipage}[t]{0.47\columnwidth}
PRÉ-CONDIÇÕES
\end{minipage} & \begin{minipage}[t]{0.47\columnwidth}
O usuário deve estar logado no sistema.
\end{minipage}
\\\hline
\begin{minipage}[t]{0.47\columnwidth}
ATORES
\end{minipage} & \begin{minipage}[t]{0.47\columnwidth}
Usuário
\end{minipage}
\\\hline
\begin{minipage}[t]{0.47\columnwidth}
DESCRIÇÃO
\end{minipage} & \begin{minipage}[t]{0.47\columnwidth}
\begin{enumerate}
\def\labelenumi{\arabic{enumi}.}
\itemsep1pt\parskip0pt\parsep0pt
\item
  Informa o percentual de cálcio desejado na CTC.
\item
  Seleciona qual fonte de cálcio será utilizada na correção.
\item
  Informa o custo médio em R\$/ha do corretivo.
\item
  Informa o percentual de PRNT do corretivo.
\item
  Informa o teor de CaO do corretivo.
\\\end{enumerate}
\end{minipage}
\\\hline
\begin{minipage}[t]{0.47\columnwidth}
ALTERNATIVAS
\end{minipage} & \begin{minipage}[t]{0.47\columnwidth}
\begin{enumerate}
\def\labelenumi{\arabic{enumi}.}
\itemsep1pt\parskip0pt\parsep0pt
\item
  Caso o usuário não saiba o teor de CaO, será utilizado o valor médio
  do corretivo.
\\\end{enumerate}
\end{minipage}
\\\hline
\begin{minipage}[t]{0.47\columnwidth}
EXCEÇÃO
\end{minipage} & \begin{minipage}[t]{0.47\columnwidth}
\begin{enumerate}
\def\labelenumi{\arabic{enumi}.}
\itemsep1pt\parskip0pt\parsep0pt
\item
  Caso o usuário informe um corretivo que não exista, uma mensagem de
  erro deverá ser exibida.
\\\end{enumerate}
\end{minipage}
\\\hline

\end{longtable}





\begin{longtable}[c]{@{}|p{4cm}|p{9cm}|@{}}
\hline
\begin{minipage}[t]{0.47\columnwidth}
\textbf{RS037}
\end{minipage} & \begin{minipage}[t]{0.47\columnwidth}
Alterar dados para a correção do cálcio e magnésio no solo
\end{minipage}
\\\hline
\begin{minipage}[t]{0.47\columnwidth}
SUMÁRIO
\end{minipage} & \begin{minipage}[t]{0.47\columnwidth}
O sistema deve apresentar ao usuário, o percentual de Cálcio na CTC
atual (antes das correções).
\end{minipage}
\\\hline
\begin{minipage}[t]{0.47\columnwidth}
PRÉ-CONDIÇÕES
\end{minipage} & \begin{minipage}[t]{0.47\columnwidth}
\begin{enumerate}
\def\labelenumi{\arabic{enumi}.}
\itemsep1pt\parskip0pt\parsep0pt
\item
  O usuário deve estar logado.
\item
  O usuário deverá acessar uma análisejá existente.
\\\end{enumerate}
\end{minipage}
\\\hline
\begin{minipage}[t]{0.47\columnwidth}
ATORES
\end{minipage} & \begin{minipage}[t]{0.47\columnwidth}
Usuário.
\end{minipage}
\\\hline
\begin{minipage}[t]{0.47\columnwidth}
DESCRIÇÃO
\end{minipage} & \begin{minipage}[t]{0.47\columnwidth}
\begin{enumerate}
\def\labelenumi{\arabic{enumi}.}
\itemsep1pt\parskip0pt\parsep0pt
\item
  Informa o percentual de cálcio desejado na CTC.
\item
  Seleciona qual fonte de cálcio será utilizada na correção.
\item
  Informa o custo médio em R\textbackslash{}\$/ha do corretivo.
\item
  Informa o percentual de PRNT do corretivo.
\item
  Informa o teor de CaO do corretivo.
\\\end{enumerate}
\end{minipage}
\\\hline
\begin{minipage}[t]{0.47\columnwidth}
ALTERNATIVAS
\end{minipage} & \begin{minipage}[t]{0.47\columnwidth}
\begin{enumerate}
\def\labelenumi{\arabic{enumi}.}
\itemsep1pt\parskip0pt\parsep0pt
\item
  Caso o usuário não saiba o teor de CaO, será utilizado o valor médio
  do corretivo.
\\\end{enumerate}
\end{minipage}
\\\hline
\begin{minipage}[t]{0.47\columnwidth}
EXCEÇÃO
\end{minipage} & \begin{minipage}[t]{0.47\columnwidth}
\begin{enumerate}
\def\labelenumi{\arabic{enumi}.}
\itemsep1pt\parskip0pt\parsep0pt
\item
  Caso o usuário informe um corretivo que não exista, uma mensagem de
  erro deverá ser exibida.
\item
  No ato da alteração, uma mensagem de aviso deverá ser exibida,
  informando ao usuário que todos os dados que dependem das informações
  relacionadas ao processo de correção do cálcio e magnésio serão
  alteradas.
\\\end{enumerate}
\end{minipage}
\\\hline

\end{longtable}

\begin{longtable}[c]{@{}|p{4cm}|p{9cm}|@{}}
\hline
\begin{minipage}[t]{0.47\columnwidth}
\textbf{RS038}
\end{minipage} & \begin{minipage}[t]{0.47\columnwidth}
Exibir o teor de cálcio atualmente no solo
\end{minipage}
\\\hline
\begin{minipage}[t]{0.47\columnwidth}
SUMÁRIO
\end{minipage} & \begin{minipage}[t]{0.47\columnwidth}
O sistema deve apresentar ao usuário, o percentual de Cálcio na CTC
atual (antes das correções)
\end{minipage}
\\\hline
\begin{minipage}[t]{0.47\columnwidth}
PRÉ-CONDIÇÕES
\end{minipage} & \begin{minipage}[t]{0.47\columnwidth}
\begin{enumerate}
\def\labelenumi{\arabic{enumi}.}
\itemsep1pt\parskip0pt\parsep0pt
\item
  O usuário deverá ter preenchido a textura do solo.
\\\end{enumerate}
\end{minipage}
\\\hline
\begin{minipage}[t]{0.47\columnwidth}
ATORES
\end{minipage} & \begin{minipage}[t]{0.47\columnwidth}
Usuário
\end{minipage}
\\\hline
\begin{minipage}[t]{0.47\columnwidth}
DESCRIÇÃO
\end{minipage} & \begin{minipage}[t]{0.47\columnwidth}
\begin{enumerate}
\def\labelenumi{\arabic{enumi}.}
\itemsep1pt\parskip0pt\parsep0pt
\item
  O sistema exibe em um input read-onlyo percentual de cálcio al no
  solo, de acordo com a textura do solo selecionada.
\\\end{enumerate}
\end{minipage}
\\\hline
\begin{minipage}[t]{0.47\columnwidth}
ALTERNATIVAS
\end{minipage} & \begin{minipage}[t]{0.47\columnwidth}
\end{minipage}
\\\hline
\begin{minipage}[t]{0.47\columnwidth}
EXCEÇÃO
\end{minipage} & \begin{minipage}[t]{0.47\columnwidth}
\end{minipage}
\\\hline

\end{longtable}

\begin{longtable}[c]{@{}|p{4cm}|p{9cm}|@{}}
\hline
\begin{minipage}[t]{0.47\columnwidth}
\textbf{RS039}
\end{minipage} & \begin{minipage}[t]{0.47\columnwidth}
Exibir o teor de cálcio ideal no solo
\end{minipage}
\\\hline
\begin{minipage}[t]{0.47\columnwidth}
SUMÁRIO
\end{minipage} & \begin{minipage}[t]{0.47\columnwidth}
O sistema deve apresentar ao usuário, o percentual ideal de Cálcio na
CTC.
\end{minipage}
\\\hline
\begin{minipage}[t]{0.47\columnwidth}
PRÉ-CONDIÇÕES
\end{minipage} & \begin{minipage}[t]{0.47\columnwidth}
\begin{enumerate}
\def\labelenumi{\arabic{enumi}.}
\itemsep1pt\parskip0pt\parsep0pt
\item
  O usuário deverá ter preenchido a textura do solo.
\item
  O usuário deverá ter preenchido a análise do solo.
\item
  O usuário deverá ter calculado o percentual de cálcio ideal na CTC.
\\\end{enumerate}
\end{minipage}
\\\hline
\begin{minipage}[t]{0.47\columnwidth}
ATORES
\end{minipage} & \begin{minipage}[t]{0.47\columnwidth}
Usuário
\end{minipage}
\\\hline
\begin{minipage}[t]{0.47\columnwidth}
DESCRIÇÃO
\end{minipage} & \begin{minipage}[t]{0.47\columnwidth}
\begin{enumerate}
\def\labelenumi{\arabic{enumi}.}
\itemsep1pt\parskip0pt\parsep0pt
\item
  O sistema exibe o percentual de cálcio ideal no solo, de acordo com a
  textura do solo selecionada.
\\\end{enumerate}
\end{minipage}
\\\hline
\begin{minipage}[t]{0.47\columnwidth}
ALTERNATIVAS
\end{minipage} & \begin{minipage}[t]{0.47\columnwidth}
\end{minipage}
\\\hline
\begin{minipage}[t]{0.47\columnwidth}
EXCEÇÃO
\end{minipage} & \begin{minipage}[t]{0.47\columnwidth}
\end{minipage}
\\\hline

\end{longtable}

\begin{longtable}[c]{@{}|p{4cm}|p{9cm}|@{}}
\hline
\begin{minipage}[t]{0.47\columnwidth}
\textbf{RS040}
\end{minipage} & \begin{minipage}[t]{0.47\columnwidth}
Calcular o percentual de cálcio ideal no solo
\end{minipage}
\\\hline
\begin{minipage}[t]{0.47\columnwidth}
SUMÁRIO
\end{minipage} & \begin{minipage}[t]{0.47\columnwidth}
O sistema deve apresentar ao usuário, o percentual ideal de Cálcio na
CTC.
\end{minipage}
\\\hline
\begin{minipage}[t]{0.47\columnwidth}
PRÉ-CONDIÇÕES
\end{minipage} & \begin{minipage}[t]{0.47\columnwidth}
\begin{enumerate}
\def\labelenumi{\arabic{enumi}.}
\itemsep1pt\parskip0pt\parsep0pt
\item
  O usuário deverá ter preenchido a textura do solo.
\item
  O usuário deverá ter preenchido a análise do solo.
\\\end{enumerate}
\end{minipage}
\\\hline
\begin{minipage}[t]{0.47\columnwidth}
ATORES
\end{minipage} & \begin{minipage}[t]{0.47\columnwidth}
Usuário
\end{minipage}
\\\hline
\begin{minipage}[t]{0.47\columnwidth}
DESCRIÇÃO
\end{minipage} & \begin{minipage}[t]{0.47\columnwidth}
\begin{enumerate}
\def\labelenumi{\arabic{enumi}.}
\itemsep1pt\parskip0pt\parsep0pt
\item
  O sistema exibe o percentual de cálcio ideal no solo, de acordo com a
  textura do solo selecionada.
\\\end{enumerate}
\end{minipage}
\\\hline
\begin{minipage}[t]{0.47\columnwidth}
ALTERNATIVAS
\end{minipage} & \begin{minipage}[t]{0.47\columnwidth}
\end{minipage}
\\\hline
\begin{minipage}[t]{0.47\columnwidth}
EXCEÇÃO
\end{minipage} & \begin{minipage}[t]{0.47\columnwidth}
\end{minipage}
\\\hline

\end{longtable}



\begin{longtable}[c]{@{}|p{4cm}|p{9cm}|@{}}
\hline
\begin{minipage}[t]{0.47\columnwidth}
\textbf{RS041}
\end{minipage} & \begin{minipage}[t]{0.47\columnwidth}
Exibir teor de magnésio atualmente no solo
\end{minipage}
\\\hline
\begin{minipage}[t]{0.47\columnwidth}
SUMÁRIO
\end{minipage} & \begin{minipage}[t]{0.47\columnwidth}
O sistema deve apresentar ao usuário o percentual de Magnésio atual na
CTC.
\end{minipage}
\\\hline
\begin{minipage}[t]{0.47\columnwidth}
PRÉ-CONDIÇÕES
\end{minipage} & \begin{minipage}[t]{0.47\columnwidth}
\begin{enumerate}
\def\labelenumi{\arabic{enumi}.}
\itemsep1pt\parskip0pt\parsep0pt
\item
  O usuário deverá ter preenchido a textura do solo.
\item
  O usuário deverá ter preenchido a análise do solo.
\\\end{enumerate}
\end{minipage}
\\\hline
\begin{minipage}[t]{0.47\columnwidth}
ATORES
\end{minipage} & \begin{minipage}[t]{0.47\columnwidth}
Usuário
\end{minipage}
\\\hline
\begin{minipage}[t]{0.47\columnwidth}
DESCRIÇÃO
\end{minipage} & \begin{minipage}[t]{0.47\columnwidth}
\begin{enumerate}
\def\labelenumi{\arabic{enumi}.}
\itemsep1pt\parskip0pt\parsep0pt
\item
  O sistema exibe o percentual de cálcio atual no solo, de acordo com a
  textura do solo selecionada.
\\\end{enumerate}
\end{minipage}
\\\hline
\begin{minipage}[t]{0.47\columnwidth}
ALTERNATIVAS
\end{minipage} & \begin{minipage}[t]{0.47\columnwidth}
\end{minipage}
\\\hline
\begin{minipage}[t]{0.47\columnwidth}
EXCEÇÃO
\end{minipage} & \begin{minipage}[t]{0.47\columnwidth}
\end{minipage}
\\\hline

\end{longtable}

\begin{longtable}[c]{@{}|p{4cm}|p{9cm}|@{}}
\hline
\begin{minipage}[t]{0.47\columnwidth}
\textbf{RS042}
\end{minipage} & \begin{minipage}[t]{0.47\columnwidth}
Exibir o percentual ideal de magnésio no solo
\end{minipage}
\\\hline
\begin{minipage}[t]{0.47\columnwidth}
SUMÁRIO
\end{minipage} & \begin{minipage}[t]{0.47\columnwidth}
O sistema deve apresentar ao usuário o percentual de Magnésio ideal na
CTC.
\end{minipage}
\\\hline
\begin{minipage}[t]{0.47\columnwidth}
PRÉ-CONDIÇÕES
\end{minipage} & \begin{minipage}[t]{0.47\columnwidth}
\begin{enumerate}
\def\labelenumi{\arabic{enumi}.}
\itemsep1pt\parskip0pt\parsep0pt
\item
  O usuário deverá ter preenchido a textura do solo.
\item
  O usuário deverá ter preenchido a análise do solo.
\\\end{enumerate}
\end{minipage}
\\\hline
\begin{minipage}[t]{0.47\columnwidth}
ATORES
\end{minipage} & \begin{minipage}[t]{0.47\columnwidth}
Usuário
\end{minipage}
\\\hline
\begin{minipage}[t]{0.47\columnwidth}
DESCRIÇÃO
\end{minipage} & \begin{minipage}[t]{0.47\columnwidth}
\begin{enumerate}
\def\labelenumi{\arabic{enumi}.}
\itemsep1pt\parskip0pt\parsep0pt
\item
  O sistema exibe o percentual de magnésio ideal no solo, de acordo com
  a textura do solo selecionada.
\\\end{enumerate}
\end{minipage}
\\\hline
\begin{minipage}[t]{0.47\columnwidth}
ALTERNATIVAS
\end{minipage} & \begin{minipage}[t]{0.47\columnwidth}
\end{minipage}
\\\hline
\begin{minipage}[t]{0.47\columnwidth}
EXCEÇÃO
\end{minipage} & \begin{minipage}[t]{0.47\columnwidth}
\end{minipage}
\\\hline
\end{longtable}



\begin{longtable}[c]{@{}|p{4cm}|p{9cm}|@{}}
\hline
\begin{minipage}[t]{0.47\columnwidth}
\textbf{RS043}
\end{minipage} & \begin{minipage}[t]{0.47\columnwidth}
Exibir o valor do magnésio no solo após as correções
\end{minipage}
\\\hline
\begin{minipage}[t]{0.47\columnwidth}
SUMÁRIO
\end{minipage} & \begin{minipage}[t]{0.47\columnwidth}
O sistema deve apresentar ao usuário o percentual de Magnésio na CTC
após as correções.
\end{minipage}
\\\hline
\begin{minipage}[t]{0.47\columnwidth}
PRÉ-CONDIÇÕES
\end{minipage} & \begin{minipage}[t]{0.47\columnwidth}
\begin{enumerate}
\def\labelenumi{\arabic{enumi}.}
\itemsep1pt\parskip0pt\parsep0pt
\item
  O usuário deverá ter preenchido a textura do solo.
\item
  O usuário deverá ter preenchido a análise do solo.
\item
  O usuário deverá ter preenchido os dados sobre a correção do cálcio e
  magnésio no solo.
\\\end{enumerate}
\end{minipage}
\\\hline
\begin{minipage}[t]{0.47\columnwidth}
ATORES
\end{minipage} & \begin{minipage}[t]{0.47\columnwidth}
Usuário
\end{minipage}
\\\hline
\begin{minipage}[t]{0.47\columnwidth}
DESCRIÇÃO
\end{minipage} & \begin{minipage}[t]{0.47\columnwidth}
\begin{enumerate}
\def\labelenumi{\arabic{enumi}.}
\itemsep1pt\parskip0pt\parsep0pt
\item
  O sistema exibe o percentual de magnésio no solo após as correções.
\\\end{enumerate}
\end{minipage}
\\\hline
\begin{minipage}[t]{0.47\columnwidth}
ALTERNATIVAS
\end{minipage} & \begin{minipage}[t]{0.47\columnwidth}
\end{minipage}
\\\hline
\begin{minipage}[t]{0.47\columnwidth}
EXCEÇÃO
\end{minipage} & \begin{minipage}[t]{0.47\columnwidth}
\end{minipage}
\\\hline

\end{longtable}

\begin{longtable}[c]{@{}|p{4cm}|p{9cm}|@{}}
\hline
\begin{minipage}[t]{0.47\columnwidth}
\textbf{RS044}
\end{minipage} & \begin{minipage}[t]{0.47\columnwidth}
Calcular o teor de magnésio no solo após as correções
\end{minipage}
\\\hline
\begin{minipage}[t]{0.47\columnwidth}
SUMÁRIO
\end{minipage} & \begin{minipage}[t]{0.47\columnwidth}
O sistema deve calcular o percentual de Magnésio na CTC após as
correções.
\end{minipage}
\\\hline
\begin{minipage}[t]{0.47\columnwidth}
PRÉ-CONDIÇÕES
\end{minipage} & \begin{minipage}[t]{0.47\columnwidth}
\begin{enumerate}
\def\labelenumi{\arabic{enumi}.}
\itemsep1pt\parskip0pt\parsep0pt
\item
  O usuário deverá ter preenchido a textura do solo.
\item
  O usuário deverá ter preenchido a análise do solo.
\item
  O usuário deverá ter preenchido a etapa de correção do cálcio e
  magnésio.
\\\end{enumerate}
\end{minipage}
\\\hline
\begin{minipage}[t]{0.47\columnwidth}
ATORES
\end{minipage} & \begin{minipage}[t]{0.47\columnwidth}
Usuário
\end{minipage}
\\\hline
\begin{minipage}[t]{0.47\columnwidth}
DESCRIÇÃO
\end{minipage} & \begin{minipage}[t]{0.47\columnwidth}
\begin{enumerate}
\def\labelenumi{\arabic{enumi}.}
\itemsep1pt\parskip0pt\parsep0pt
\item
  O sistema calcula o valor do magnésio no solo após as correções.
\\\end{enumerate}
\end{minipage}
\\\hline
\begin{minipage}[t]{0.47\columnwidth}
ALTERNATIVAS
\end{minipage} & \begin{minipage}[t]{0.47\columnwidth}
\end{minipage}
\\\hline
\begin{minipage}[t]{0.47\columnwidth}
EXCEÇÃO
\end{minipage} & \begin{minipage}[t]{0.47\columnwidth}
\end{minipage}
\\\hline

\end{longtable}



\begin{longtable}[c]{@{}|p{4cm}|p{9cm}|@{}}
\hline
\begin{minipage}[t]{0.47\columnwidth}
\textbf{RS045}
\end{minipage} & \begin{minipage}[t]{0.47\columnwidth}
Exibir a quantidade de corretivo de cálcio e magnésio a ser aplicada no
solo.
\end{minipage}
\\\hline
\begin{minipage}[t]{0.47\columnwidth}
SUMÁRIO
\end{minipage} & \begin{minipage}[t]{0.47\columnwidth}
O sistema deve exibir a quantidade a ser aplicada, em toneladas por
hectare, do corretivo informado.
\end{minipage}
\\\hline
\begin{minipage}[t]{0.47\columnwidth}
PRÉ-CONDIÇÕES
\end{minipage} & \begin{minipage}[t]{0.47\columnwidth}
\begin{enumerate}
\def\labelenumi{\arabic{enumi}.}
\itemsep1pt\parskip0pt\parsep0pt
\item
  O usuário deverá ter calculado a quantidade de corretivo a ser
  aplicada no solo.
\\\end{enumerate}
\end{minipage}
\\\hline
\begin{minipage}[t]{0.47\columnwidth}
ATORES
\end{minipage} & \begin{minipage}[t]{0.47\columnwidth}
Usuário
\end{minipage}
\\\hline
\begin{minipage}[t]{0.47\columnwidth}
DESCRIÇÃO
\end{minipage} & \begin{minipage}[t]{0.47\columnwidth}
\begin{enumerate}
\def\labelenumi{\arabic{enumi}.}
\itemsep1pt\parskip0pt\parsep0pt
\item
  O sistema exibe a quantidade em toneladas por hectare (ton/ha) de
  corretivo a ser utilizada na correção do cálcio e magnésio do solo.
\\\end{enumerate}
\end{minipage}
\\\hline
\begin{minipage}[t]{0.47\columnwidth}
ALTERNATIVAS
\end{minipage} & \begin{minipage}[t]{0.47\columnwidth}
\end{minipage}
\\\hline
\begin{minipage}[t]{0.47\columnwidth}
EXCEÇÃO
\end{minipage} & \begin{minipage}[t]{0.47\columnwidth}
\end{minipage}
\\\hline

\end{longtable}

\begin{longtable}[c]{@{}|p{4cm}|p{9cm}|@{}}
\hline
\begin{minipage}[t]{0.47\columnwidth}
\textbf{RS046}
\end{minipage} & \begin{minipage}[t]{0.47\columnwidth}
Calcular a quantidade de corretivo a ser aplicada no solo.
\end{minipage}
\\\hline
\begin{minipage}[t]{0.47\columnwidth}
SUMÁRIO
\end{minipage} & \begin{minipage}[t]{0.47\columnwidth}
O sistema deve calcular a quantidade a ser aplicada, em toneladas por
hectare, do corretivo informado.
\end{minipage}
\\\hline
\begin{minipage}[t]{0.47\columnwidth}
PRÉ-CONDIÇÕES
\end{minipage} & \begin{minipage}[t]{0.47\columnwidth}
\begin{enumerate}
\def\labelenumi{\arabic{enumi}.}
\itemsep1pt\parskip0pt\parsep0pt
\item
  O usuário deverá ter preenchido a textura do solo.
\item
  O usuário deverá ter preenchido a análise do solo.
\item
  O usuário deverá ter preenchido a etapa de correção do cálcio e
  magnésio.
\item
  O usuário deverá ter informado as informações da propriedade.
\\\end{enumerate}
\end{minipage}
\\\hline
\begin{minipage}[t]{0.47\columnwidth}
ATORES
\end{minipage} & \begin{minipage}[t]{0.47\columnwidth}
Usuário
\end{minipage}
\\\hline
\begin{minipage}[t]{0.47\columnwidth}
DESCRIÇÃO
\end{minipage} & \begin{minipage}[t]{0.47\columnwidth}
\begin{enumerate}
\def\labelenumi{\arabic{enumi}.}
\itemsep1pt\parskip0pt\parsep0pt
\item
  O sistema calcula a quantidade de corretivo a ser utilizada.
\\\end{enumerate}
\end{minipage}
\\\hline
\begin{minipage}[t]{0.47\columnwidth}
ALTERNATIVAS
\end{minipage} & \begin{minipage}[t]{0.47\columnwidth}
\end{minipage}
\\\hline
\begin{minipage}[t]{0.47\columnwidth}
EXCEÇÃO
\end{minipage} & \begin{minipage}[t]{0.47\columnwidth}
\end{minipage}
\\\hline

\end{longtable}



\begin{longtable}[c]{@{}|p{4cm}|p{9cm}|@{}}
\hline
\begin{minipage}[t]{0.47\columnwidth}
\textbf{RS047}
\end{minipage} & \begin{minipage}[t]{0.47\columnwidth}
Exibir a quantidade de enxofre que o corretivo fornecerá à terra
\end{minipage}
\\\hline
\begin{minipage}[t]{0.47\columnwidth}
SUMÁRIO
\end{minipage} & \begin{minipage}[t]{0.47\columnwidth}
O sistema deve exibir o valor, em quilogramas por hectare, da quantidade
de enxofre que o corretivo fornecerá.
\end{minipage}
\\\hline
\begin{minipage}[t]{0.47\columnwidth}
PRÉ-CONDIÇÕES
\end{minipage} & \begin{minipage}[t]{0.47\columnwidth}
\begin{enumerate}
\def\labelenumi{\arabic{enumi}.}
\itemsep1pt\parskip0pt\parsep0pt
\item
  O usuário deverá ter preenchido a textura do solo.
\item
  O usuário deverá ter preenchido a análise do solo.
\item
  O usuário deverá ter preenchido a etapa de correção do cálcio e
  magnésio.
\\\end{enumerate}
\end{minipage}
\\\hline
\begin{minipage}[t]{0.47\columnwidth}
ATORES
\end{minipage} & \begin{minipage}[t]{0.47\columnwidth}
Usuário
\end{minipage}
\\\hline
\begin{minipage}[t]{0.47\columnwidth}
DESCRIÇÃO
\end{minipage} & \begin{minipage}[t]{0.47\columnwidth}
\begin{enumerate}
\def\labelenumi{\arabic{enumi}.}
\itemsep1pt\parskip0pt\parsep0pt
\item
  O sistema calcula a quantidade de enxofre que o corretivo fornecerá à
  terra.
\item
  Exibe o valor para o usuário na tela de correção/equilíbrio do cálcio
  e magnésio.
\\\end{enumerate}
\end{minipage}
\\\hline
\begin{minipage}[t]{0.47\columnwidth}
ALTERNATIVAS
\end{minipage} & \begin{minipage}[t]{0.47\columnwidth}
\begin{enumerate}
\def\labelenumi{\arabic{enumi}.}
\itemsep1pt\parskip0pt\parsep0pt
\item
  Quando a quantidade for igual a 0, o campo pode ficar oculto.
\\\end{enumerate}
\end{minipage}
\\\hline
\begin{minipage}[t]{0.47\columnwidth}
EXCEÇÃO
\end{minipage} & \begin{minipage}[t]{0.47\columnwidth}
\end{minipage}
\\\hline

\end{longtable}



\begin{longtable}[c]{@{}|p{4cm}|p{9cm}|@{}}
\hline
\begin{minipage}[t]{0.47\columnwidth}
\textbf{RS048}
\end{minipage} & \begin{minipage}[t]{0.47\columnwidth}
Exibir a quantidade de enxofre necessária
\end{minipage}
\\\hline
\begin{minipage}[t]{0.47\columnwidth}
SUMÁRIO
\end{minipage} & \begin{minipage}[t]{0.47\columnwidth}
O sistema deve exibir a quantidade suficiente, em quilogramas por
hectare, da quantidade de enxofre necessária.
\end{minipage}
\\\hline
\begin{minipage}[t]{0.47\columnwidth}
PRÉ-CONDIÇÕES
\end{minipage} & \begin{minipage}[t]{0.47\columnwidth}
\begin{enumerate}
\def\labelenumi{\arabic{enumi}.}
\itemsep1pt\parskip0pt\parsep0pt
\item
  O usuário deverá ter preenchido a textura do solo.
\item
  O usuário deverá ter preenchido a análise do solo.
\item
  O usuário deverá ter preenchido a fonte de corretivo de cálcio e
  magnésio.
\\\end{enumerate}
\end{minipage}
\\\hline
\begin{minipage}[t]{0.47\columnwidth}
ATORES
\end{minipage} & \begin{minipage}[t]{0.47\columnwidth}
Usuário
\end{minipage}
\\\hline
\begin{minipage}[t]{0.47\columnwidth}
DESCRIÇÃO
\end{minipage} & \begin{minipage}[t]{0.47\columnwidth}
\begin{enumerate}
\def\labelenumi{\arabic{enumi}.}
\itemsep1pt\parskip0pt\parsep0pt
\item
  O sistema exibe a quantidade de enxofre necessária para a correção do
  cálcio e magnésio no solo.
\\\end{enumerate}
\end{minipage}
\\\hline
\begin{minipage}[t]{0.47\columnwidth}
ALTERNATIVAS
\end{minipage} & \begin{minipage}[t]{0.47\columnwidth}
\begin{enumerate}
\def\labelenumi{\arabic{enumi}.}
\itemsep1pt\parskip0pt\parsep0pt
\item
  Só será exibido caso a quantidade de enxofre seja maior que 0.
\\\end{enumerate}
\end{minipage}
\\\hline
\begin{minipage}[t]{0.47\columnwidth}
EXCEÇÃO
\end{minipage} & \begin{minipage}[t]{0.47\columnwidth}
\end{minipage}
\\\hline

\end{longtable}



\begin{longtable}[c]{@{}|p{4cm}|p{9cm}|@{}}
\hline
\begin{minipage}[t]{0.47\columnwidth}
\textbf{RS049}
\end{minipage} & \begin{minipage}[t]{0.47\columnwidth}
Calcular e exibir o valor de V\% atual, ideal e após as correções
\end{minipage}
\\\hline
\begin{minipage}[t]{0.47\columnwidth}
SUMÁRIO
\end{minipage} & \begin{minipage}[t]{0.47\columnwidth}
O sistema deve calcular e exibir os valores de V\% atual, ideal e após
as correções.
\end{minipage}
\\\hline
\begin{minipage}[t]{0.47\columnwidth}
PRÉ-CONDIÇÕES
\end{minipage} & \begin{minipage}[t]{0.47\columnwidth}
\begin{enumerate}
\def\labelenumi{\arabic{enumi}.}
\itemsep1pt\parskip0pt\parsep0pt
\item
  O usuário deve estar logado.
\item
  A etapa de preenchimento dos dados da propriedade deverá estar
  preenchida.
\item
  A etapa de preenchimento da análise do solo deverá estar preenchida.
\item
  A etapa de preenchimento da matéria orgânica deverá estar preenchida.
\item
  A etapa de preenchimento da correção do fósforo deverá estar
  preenchida.
\item
  A etapa de preenchimento da correção do potássio deverá estar
  preenchida.
\item
  A etapa de preenchimento da correção do cálcio e magnésio deverá estar
  preenchida.
\\\end{enumerate}
\end{minipage}
\\\hline
\begin{minipage}[t]{0.47\columnwidth}
ATORES
\end{minipage} & \begin{minipage}[t]{0.47\columnwidth}
Usuário
\end{minipage}
\\\hline
\begin{minipage}[t]{0.47\columnwidth}
DESCRIÇÃO
\end{minipage} & \begin{minipage}[t]{0.47\columnwidth}
\begin{enumerate}
\def\labelenumi{\arabic{enumi}.}
\itemsep1pt\parskip0pt\parsep0pt
\item
  O sistema exibe o valor de V\% atual.
\item
  O sistema exibe o valor de V\% após as correções.
\item
  O sistema exibe o valor de V\% ideal.
\\\end{enumerate}
\end{minipage}
\\\hline
\begin{minipage}[t]{0.47\columnwidth}
ALTERNATIVAS
\end{minipage} & \begin{minipage}[t]{0.47\columnwidth}
\end{minipage}
\\\hline
\begin{minipage}[t]{0.47\columnwidth}
EXCEÇÃO
\end{minipage} & \begin{minipage}[t]{0.47\columnwidth}
\end{minipage}
\\\hline

\end{longtable}

\endgroup


\clearpage

\section{Requisitos Não Funcionais}
\label{sec:titSecReqNaoFunc}

Texto...

% Please add the following required packages to your document preamble:
% \usepackage{longtable}
% Note: It may be necessary to compile the document several times to get a multi-page table to line up properly
\begin{longtable}{|p{1.5cm}|p{3cm}|p{7cm}|p{2.5cm}|}
    \hline
    ID     & REQUISITO                               & DESCRIÇÃO                                                                          & PRIORIDADE    \\\hline
    \endfirsthead
    %
    \multicolumn{4}{c}%
    {{\bfseries Continuação da tabela \thetable\ da página anterior}}                                                                                     \\\hline
    ID     & REQUISITO                               & DESCRIÇÃO                                                                          & PRIORIDADE    \\\hline
    \endhead
    %
    RNF001 & Adaptar em diferentes formatos de tela. & O sistema deve ser adaptado em diferentes tamanhos de tela. (Requisito de exemplo) & Adptabilidade \\\hline
\end{longtable}
\clearpage

\section{Casos de Uso}
\label{sec:titSecCasoUso}

Segundo \cite{guedes2018uml}, o diagrama de casos de uso "apresenta uma linguagem simples e de fácil compreensão para que os usuários possam ter uma ideia geral de como o sistema irá se comportar. Devido a sua simplicidade,

\begin{figure}[H]
    \centering
    \includegraphics[width=13cm]{./dados/analise/casouso.jpg}
    \caption{Diagrama de Caso de Uso Geral}
    \label{fig:diagramaCasoUso}
\end{figure}

Na \autoref{fig:diagramaCasoUso}
\clearpage

\section{Diagrama de Classes}
\label{sec:titSecDiagClasse}

O Diagrama de Classes, segundo \cite{guedes2018uml}, define a estrutura das classes utilizadas pelo sistema, determinando os atributos e métodos que cada classe tem, além de estabelecer como as classes se relacionam e trocam informações entre si. A aplicação desse diagrama no sitema em questão pode ser vista na \autoref{fig:diagramaClasse}.

\begin{figure}[H]
    \centering
    \includegraphics[width=13cm]{./dados/analise/diagramaclasse.jpg}
    \caption{Diagrama de Classes}
    \label{fig:diagramaClasse}
\end{figure}
\clearpage

\subsection{Projeto de Banco de Dados}
\label{sec:titSecBancoDados}

Segundo \citeonline{silberschatz2006sistema}, o objetivo do projeto de Banco de Dados é a construção de um conjunto de estruturas que permitam a representação de um dado de forma não redundante e que esse dado possa ser recuperado de forma simples. Este processo geralmente passa em diferentes níveis de abstração. Primeiramente, inicia-se com o entendimento do problema, para depois seguir por diferentes fases de modelagem, até chegar à implementação. Nesse processo, um modelo considerado padrão para a etapa inicial de modelagem é o entidade-relacionamento (ER) \cite{heuser2009projeto}.

\subsubsection{Diagrama de Entidade-Relacionamento}
\label{sec:titSecDiagER}

A modelagem entidade-relacionamento é a mais utilizada e difundida técnica de modelagem de dados. Nesta técnica, o modelo de dados é representado por meio de um modelo entidade-relacionamento (modelo ER). Comumente, o modelo ER é representado graficamente por meio de um diagrama entidade-relacionamento (DER) \cite{heuser2009projeto}.

Essa modelagem fornece uma visão acerca dos atributos, tabelas e relacionamentos que o banco de dados terá. A \autoref{fig:diagramaer} representa o DER do projetado neste documento, utilizando a notação pé de galinha.

\begin{figure}[H]
    \centering
    \caption{Diagrama Entidade-Relacionamento}
    \includegraphics[width=13cm]{dados/figuras/diagramaer.png}
    \label{fig:diagramaer}
    \fonte{Autoria própria}
\end{figure}

\subsubsection{Dicionário de Dados}
\label{sec:titSecDiagERDicionario}

Nas tabelas contidas nessa subseção, é possível entender a organização dos dados e as informações que serão armazenadas.

\begin{landscape}
    \subsubsubsection{Tabela de Propriedades}
    \label{sec:titSubSecPropriedades}

    \begin{table}[H]
        \centering
        \caption[Tabela propriedades]{Na tabela propriedades ficarão armazenados os dados referentes à propriedade.
            \label{tab:tabela-er-propriedades}}
        \begin{tabular}{|p{4cm}|p{3cm}|p{2cm}|p{1cm}|p{2cm}|p{8cm}|}
            \hline
            Atributo     & Classe       & Domínio  & Bytes & Restrição & Descrição            \\\hline
            id           & Determinante & Numérico & 4     & PK, AI    &                      \\\hline
            proprietario & Simples      & Texto    & 255   & NN        & Nome do proprietário \\\hline
            municipio    & Simples      & Texto    & 255   & NN        & e.g. Dois Vizinhos   \\\hline
            estado       & Simples      & Texto    & 2     & NN        & e.g. PR              \\\hline
            lote         & Simples      & Texto    & 20    & NN        & e.g Q-103            \\\hline
            area         & Simples      & Numérico & 255   & NN        & Em metros quadrados  \\\hline
        \end{tabular}
    \end{table}

    \subsubsubsection{Tabela de Talhões}
    \label{sec:titSubSecTalhoes}

    \begin{table}[H]
        \centering
        \caption[Tabela talhões]{Na tabela talhões serão armazenados os dados referentes aos talhões da propriedade.
            \label{tab:tabela-er-talhoes}}
        \begin{tabular}{|p{4cm}|p{3cm}|p{2cm}|p{1cm}|p{2cm}|p{8cm}|}
            \hline
            Atributo              & Classe       & Domínio    & Bytes & Restrição & Descrição                        \\\hline
            id                    & Determinante & Numérico   & 4     & PK, AI    &                                  \\\hline
            area                  & Simples      & Numérico   & 10    & NN        & Em metros quadrados              \\\hline
            propriedades\_id      & Composto     & Numérico   & 255   & FK, NN    & Referência à tabela propriedades \\\hline
            plantio               & Simples      & Enumeravel & 1     & NN        & e.g. 1 (Plantio Direto)          \\\hline
            textura               & Simples      & Enumeravel & 1     & NN        & e.g. 2 (Textura Média)           \\\hline
            responsavel\_tecnico  & Simples      & Texto      & 255   & NN        & e.g. Pedro Cecere Filho          \\\hline
            profundidade\_amostra & Simples      & Numérico   & 255   &           & Em centímetros                   \\\hline
        \end{tabular}
    \end{table}

    \subsubsubsection{Tabela de Análises}
    \label{sec:titSubSecAnalises}

    \begin{table}[H]
        \centering
        \caption[Tabela análises]{Na tabela análises serão armazenados os dados do laudo técnico referente ao talhão.
            \label{tab:tabela-er-analises}}
        \begin{tabular}{|p{4cm}|p{3cm}|p{2cm}|p{1cm}|p{2cm}|p{8cm}|}
            \hline
            Atributo             & Classe       & Domínio  & Bytes & Restrição & Descrição                          \\\hline
            fosforo              & Simples      & Numérico & 4     & NN        & centimol de carga/decimetro cúbico \\\hline
            potassio             & Simples      & Numérico & 4     & NN        & centimol de carga/decimetro cúbico \\\hline
            calcio               & Simples      & Numérico & 4     & NN        & centimol de carga/decimetro cúbico \\\hline
            magnesio             & Simples      & Numérico & 4     & NN        & centimol de carga/decimetro cúbico \\\hline
            enxofre              & Simples      & Numérico & 4     & NN        & centimol de carga/decimetro cúbico \\\hline
            aluminio             & Simples      & Numérico & 4     & NN        & centimol de carga/decimetro cúbico \\\hline
            hidrogenio\_aluminio & Simples      & Numérico & 4     & NN        & centimol de carga/decimetro cúbico \\\hline
            materia\_organica    & Simples      & Numérico & 4     & NN        & centimol de carga/decimetro cúbico \\\hline
            etapa                & Determinante & Numérico & 4     & PK, AI    &                                    \\\hline
            talhoes\_id          & Composto     & Numérico & 4     & PK, FK    & Referência à tabela talhoes        \\\hline
        \end{tabular}
    \end{table}

    \subsubsubsection{Tabela de Correções}
    \label{sec:titSubSecCorrecoes}

    \begin{table}[H]
        \centering
        \caption[Tabela correções]{Na tabela Correções serão armazenados os dados úteis para o cálculo do equilíbrio e correção do solo.
            \label{tab:tabela-er-correcoes}}
        \begin{tabular}{|p{4cm}|p{3cm}|p{2cm}|p{1cm}|p{2cm}|p{8cm}|}
            \hline
            Atributo                    & Classe       & Domínio  & Bytes & Restrição & Descrição                         \\\hline
            id                          & Determinante & Numérico & 4     & PK, AI    &                                   \\\hline
            teor\_fosforo\_desejado     & Simples      & Numérico & 1     & NN        & em percentual                     \\\hline
            fonte\_de\_fosforo          & Composto     & Numérico & 4     & NN        & em referência à tabela corretivos \\\hline
            eficiencia\_fosforo         & Simples      & Numérico & 1     & NN        & em percentual                     \\\hline
            custo\_ton\_fonte\_fosforo  & Simples      & Numérico & 8     & NN        & em R\$/tonelada                   \\\hline
            fonte\_de\_potassio         & Composto     & Numérico & 4     & NN        & em referência à tabela corretivos \\\hline
            fonte\_de\_calcio\_magnesio & Composto     & Numérico & 4     & NN        & em referência à tabela corretivos \\\hline
            teor\_potassio\_desejado    & Simples      & Numérico & 1     & NN        & em percentual                     \\\hline
            custo\_ton\_fonte\_potassio & Simples      & Numérico & 8     & NN        & em R\$/tonelada                   \\\hline
            teor\_calcio\_desejado      & Simples      & Numérico & 1     & NN        & em percentual                     \\\hline
            prnt\_fonte\_de\_calcio     & Simples      & Numérico & 1     & NN        & em percentual                     \\\hline
            teor\_cao\_font\_de\_calcio & Simples      & Numérico & 1     & NN        & em percentual                     \\\hline
            custo\_ton\_fonte\_calcio   & Simples      & Numérico & 8     & NN        & em R\$/tonelada                   \\\hline
            analises\_etapa             & Composto     & Numérico & 4     & FK        & Referência à tabela talhoes       \\\hline
            analises\_talhoes\_id       & Composto     & Numérico & 4     & FK        & Referência à tabela talhoes       \\\hline
        \end{tabular}
    \end{table}

    \subsubsubsection{Tabela de Corretivos}
    \label{sec:titSubSecCorretivos}

    \begin{table}[H]
        \centering
        \caption[Tabela corretivos]{Na tabela Corretivos serão armazenados os dados padrões acerca dos corretivos.
            \label{tab:tabela-er-corretivos}}
        \begin{tabular}{|p{4cm}|p{3cm}|p{2cm}|p{1cm}|p{2cm}|p{8cm}|}
            \hline
            Atributo            & Classe       & Domínio  & Bytes & Restrição & Descrição       \\\hline
            id                  & Determinante & Numérico & 4     & PK, AI    &                 \\\hline
            nome                & Simples      & Texto    & 255   & NN        & em percentual   \\\hline
            custo\_ton          & Simples      & Numérico & 8     &           & em R\$/tonelada \\\hline
            teor\_cao           & Simples      & Numérico & 1     &           & em percentual   \\\hline
            eficiencia\_fosforo & Simples      & Numérico & 1     &           & em percentual   \\\hline
            prnt                & Simples      & Numérico & 1     &           & em percentual   \\\hline
        \end{tabular}
    \end{table}

\end{landscape}
\clearpage

\section{Diagrama de Implantação}
\label{sec:titSecDiagDeploy}

O Diagrama de Implantação \cite[determina as necessidades de \textit{hardware} do sistema, as características físicas como servidores, estações, topologias e protocolos de comunicação]{guedes2018uml}, destacando todos as necessidades físicas no qual o sistema deverá ser executado. Também é mostrado como será a distribuição dos módulos do sistema nos momentos onde a aplicação estará sendo executada em diferentes servidores.

\begin{figure}[H]
    \centering
    \includegraphics[width=13cm]{./dados/analise/diagramadeploy.jpg}
    \caption{Representação do Diagrama de Implantação do Sistema}
    \label{fig:diagramaDeploy}
\end{figure}

A \autoref{fig:diagramaDeploy} mostra a aplicação desse conceito no projeto em questão. Nela é possível observar a distribuição do sistema em dois nós principais. O primeiro é o servidor de aplicação, que ecapsula o servidor Apache. Nesse mesmo servidor é possível observar a presença de um servidor de banco de dados MySQL, o qual será responsável por armazenar os dados do sistema.
O outro nó, representa o lado do cliente da aplicação, responsável pelas chamadas aos \textit{endpoints} da API Rest contida no servidor \textit{web}.

\clearpage

\section{PROTÓTIPOS DE TELA}
\label{sec:titSecPrototipos}

Os protótipos de tela são uma versão inicial de um sistema. Eles são usados, dentre outras razões, para descobrir mais sobre um problema e suas possíveis soluções. O desenvolvimento rápido e iterativo do protótipo previne gastos desnecessários e custos controlados e os stakeholders podem fazer usos pontuais de partes do sistema desde o início do desenvolvimento \citeonline{Sommerville10}.
Além disso, os protótipos podem ajudar na antecipação de mudanças que podem ser requisitadas.
Para o presente projeto foram desenvolvidas três interfaces para dois tamanhos diferentes de tela, simulando um dispositivo de tela maior (desktop, tablet) e outro para telas menores (smartphones).

O modelo de prototipação é ideal quando o cliente não tem os requisitos muito bem definidos, que é o caso do aplicativo estudado neste documento. Visto que os requisitos foram levantados com base em uma planilha, confome explicado anteriormente na \autoref{sub:processo}.

Nas próximas subseções serão exibidos os protótipos de tela para três principais operações que o usuário poderá realizar no sistema, que são a tela que o usuário é enviado após fazer o login, a tela para criação de um novo recurso no sistema e a tela de exibição de um recurso.

\subsection{Protótipos para tela inicial do sistema}
\label{sec:titSecPrototiposHome}

As Figuras \ref{fig:prototipo_home_desk} e \ref{fig:prototipo_home_mobile}, representam a visão da página inicial do sistema, que será exibida após o login do usuário. É possível observar que logo na parte superior, existirão quadros informativos sobre a quantidade de propriedades assistidas, quantidade de cidades atendidas e a área em hectares que foram tratadas (ou corrigidas). Logo abaixo serão exibidas as últimas correções realizadas pelo usuário.

\begin{figure}[H]
    \centering
    \caption{Página que será exibida após o login do usuário na visão de um dispositivo desktop}
    \includegraphics[width=13cm]{./dados/figuras/prototipos/home_desktop.png}
    \label{fig:prototipo_home_desk}
    \fonte{Autoria própria}
\end{figure}

\begin{figure}[H]
    \centering
    \caption{Página que será exibida após o login do usuário na visão de um dispositivo mobile}
    \includegraphics[width=6cm]{./dados/figuras/prototipos/home_mobile.png}
    \label{fig:prototipo_home_mobile}
    \fonte{Autoria própria}
\end{figure}

\subsection{Protótipos para tela de cadastro de informação}
\label{sec:titSecPrototiposCreate}

A tela de cadastro de informações foi pensada para ser intuitiva e reativa como a planilha. Como é um formulário grande, será divido em \textit{steps} que serão exibidos na parte superior do formulário, tendo a descrição do que deve ser feito naquele \textit{step} logo abaixo. Já o formulário propriamente dito terá uma interface reativa, que lembra a planilha.

No exemplo mostrado nas Figuras \ref{fig:prototipo_create_desk} e \ref{fig:prototipo_create_mobile}, é possível observar que caso o valor de um campo esteja no intervalo ideal, este deverá ter a sua label e o contorno destacados em verde. Enquanto se o valor estiver fora do intervalo, em âmbar. 

\begin{figure}[H]
    \centering
    \caption{Criação de recurso na visão de um dispositivo desktop}
    \includegraphics[width=13cm]{./dados/figuras/prototipos/create_desktop.png}
    \label{fig:prototipo_create_desk}
    \fonte{Autoria própria}
\end{figure}

\begin{figure}[H]
    \centering
    \caption{Criação de recurso na visão de um dispositivo mobile}
    \includegraphics[width=6cm]{./dados/figuras/prototipos/create_mobile.png}
    \label{fig:prototipo_create_mobile}
    \fonte{Autoria própria}
\end{figure}


\subsection{Protótipos para tela de exibição de informação}
\label{sec:titSecPrototiposShow}

A tela de exibição da informação também seguirá o modelo de \textit{steps}. Ela deverá ser dividia em 3 colunas para representar os três estados da correção do solo: naquele momento, desejado e após as correções. É possível notar que os textos deverão estar coloridos de acordo com o caso no qual se encaixa. Acima do valor, amarelo. Abaixo do valor, vermelho. No intervalo, verde.

\begin{figure}[H]
    \centering
    \caption{Visualização de recurso na visão de um dispositivo desktop}
    \includegraphics[width=13cm]{./dados/figuras/prototipos/show_desktop.png}
    \label{fig:prototipo_show_desk}
    \fonte{Autoria própria}
\end{figure}

\begin{figure}[H]
    \centering
    \caption{Visualização de recurso na visão de um dispositivo mobile}
    \includegraphics[width=6cm]{./dados/figuras/prototipos/show_mobile.png}
    \label{fig:prototipo_show_mobile}
    \fonte{Autoria própria}
\end{figure}
\clearpage


\chapter{CRONOGRAMA DO PROJETO}

% \section{CRONOGRAMA DO PROJETO}
% \label{sec:cronograma}

O gráfico do abaixo demonstra a distribuição das atividades do projeto em relação ao tempo. A data da primeira reunião com o professor orientador foi considerada a data inicial do cronograma, na qual foi definido o tema a ser desenvolvido. A duração de cada atividade está na unidade de dias.

\begin{landscape}

    \begin{ganttchart}
        [
            y unit title=0.4cm,
            y unit chart=0.5cm,
            vgrid,hgrid,
            title height=1,
            bar/.style={draw,fill=green},
            bar incomplete/.append style={fill=yellow!50},
            bar height=0.7
        ]{1}{36}

        \gantttitle{Junho}{6}
        \gantttitle{Julho}{5}
        \gantttitle{Agosto}{5}
        \gantttitle{Setembro}{5}
        \gantttitle{Outubro}{5}
        \gantttitle{Novembro}{5}
        \gantttitle{Dezembro}{5} \\

        \gantttitlelist{1,...,6}{1}
        \gantttitlelist{1,...,5}{1}
        \gantttitlelist{1,...,5}{1}
        \gantttitlelist{1,...,5}{1}
        \gantttitlelist{1,...,5}{1}
        \gantttitlelist{1,...,5}{1}
        \gantttitlelist{1,...,5}{1} \\

        \gantttitlelist{1,...,36}{1} \\

        \ganttbar{Desenvolvimento da Interfaces}{2}{17} \\

        \ganttbar{Desenvolvimento da API Rest}{4}{20} \\

        \ganttbar{Correções no Documento}{5}{5}
        \ganttbar{}{10}{10}
        \ganttbar{}{15}{15}
        \ganttbar{}{20}{20}
        \ganttbar{}{24}{30}\\

        \ganttbar{Entrega do Documento}{31}{30} \\

        \ganttbar{Validação pelo técnico}{5}{4}
        \ganttbar{}{10}{9}
        \ganttbar{}{15}{14}
        \ganttbar{}{20}{19} \\

        \ganttbar{Entrega parcial}{7}{6}
        \ganttbar{}{12}{11}
        \ganttbar{}{17}{16}
        \ganttbar{}{22}{21} \\

        \ganttbar{Implantação}{22}{24} \\

        \ganttbar{Testes exploratórios pelo cliente}{25}{29} \\
        \ganttbar{Correção de bugs}{25}{29} \\

    \end{ganttchart}

\end{landscape}