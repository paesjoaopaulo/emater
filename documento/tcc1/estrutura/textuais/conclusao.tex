% CONCLUSÃO--------------------------------------------------------------------

\chapter{CONCLUSÃO}
\label{chap:conclusao}

A solução para os problemas encontrados no uso de planilhas para o cálculo de correção do solo serão implementados na aplicação \textit{web} proposta no presente trabalho. Será utilizado o método de desenvolvimento TDD para criar uma aplicação confiável e mitigar equívocos de programação e integração das partes que serão desenvolvidas durante o processo.

Para criar uma interface com características desejáveis da planilha como a reatividade à entradas de dados, será utilizado o React e o Laravel será utilizado para a implementação do \textit{backend} em PHP, responsável pelo processamento de dados da aplicação.

Pretende-se com a implantação da aplicação reduzir o tempo de geração do relatório de necessidade de correção do solo, agilizando o processo de tomada de decisão para recuperação do equilíbrio dos nutrientes do solo, fazendo com que os vegetais se desenvolvam como o desejado e os agricultores não tenham perca de produtividade em suas lavouras, gerando mais renda e conseguinte melhora na qualidade de vida dessas pessoas.

% TCC 2
%\section{TRABALHOS FUTUROS}
%\label{sec:trabalhosFuturos}

%\section{CONSIDERAÇÕES FINAIS}
%\label{sec:consideracoesFinais}

%Encerramento do trabalho acadêmico.
