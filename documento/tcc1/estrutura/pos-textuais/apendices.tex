% APÊNDICES--------------------------------------------------------------------

\begin{apendicesenv}
\partapendices

% Primeiro apêndice------------------------------------------------------------
\chapter{Nome do apêndice} % Edite para alterar o título deste apêndice
\label{chap:apendiceA}

Lembre-se que a diferença entre apêndice e anexo diz respeito à autoria do texto e/ou material ali colocado.

Caso o material ou texto suplementar ou complementar seja de sua autoria, então ele deverá ser colocado como um apêndice. Porém, caso a autoria seja de terceiros, então o material ou texto deverá ser colocado como anexo.

Caso seja conveniente, podem ser criados outros apêndices para o seu trabalho acadêmico. Basta recortar e colar este trecho neste mesmo documento. Lembre-se de alterar o "label"{} do apêndice.

Não é aconselhável colocar tudo que é complementar em um único apêndice. Organize os apêndices de modo que, em cada um deles, haja um único tipo de conteúdo. Isso facilita a leitura e compreensão para o leitor do trabalho.

% Novo apêndice----------------------------------------------------------------
\chapter{Nome do outro apêndice}
\label{chap:apendiceB}

conteúdo do novo apêndice

\end{apendicesenv}
